% !TeX root = ./main.tex
\documentclass[screen, aspectratio=43]{beamer}
\usepackage[T1]{fontenc}
\usepackage[utf8]{inputenc}

\usetheme[style=ntnu,language=bm]{ntnu2017}

\usepackage[english]{babel}
\usepackage{pythonhighlight}
\usepackage{geometry}
\usepackage{tikz}
\usetikzlibrary{positioning}
\usetikzlibrary{arrows}
\usetikzlibrary{shapes.multipart}


\title[Kræsjkurs]{Kræsjkurs} 
\subtitle{Python} 
\author[M. E. Gerhardsen]{Martin Eek Gerhardsen} 
\institute[NTNU]{Institutt for Teknisk Kybernetikk, NTNU} 
\date{2. desember 2019} 

\begin{document}

\begin{frame} 
\titlepage
\end{frame} 

\begin{frame}
    \frametitle{Plan}

    \begin{itemize}
        \item Grunnleggende
        \item Visualisering
        \item Gjennomgang av gamle eksamensoppgaver 
    \end{itemize}

\end{frame}

\begin{frame}
    \frametitle{Plan}

    \begin{enumerate}
        \item Tilordningsoperator
        \item Aritmetiske operatorer
        \item Sammenligningsoperatorer
        \item Datatyper
        \item Input og output fra bruker
        \item Funksjoner 
        \item Lister og tupler
        \item Løkker 
        \item Slicing 
        \item Dictionaries og mengder 
        \item Input og output fra filer 
        \item Unntaksbehandling
        \item Eksamensoppgaver
    \end{enumerate}

\end{frame}


\begin{frame}
    \frametitle{Tilordningsoperatoren}

    Gitt to variabler $a$ og $b$, bytt om innholdet. 

    Tilordningsoperatoren = 

    Original verdi blir overskrevet. 

\end{frame}


\begin{frame}[fragile] 
    \frametitle{Tilordningsoperatoren} 
\begin{python}
a = "hei"
b = 3.14
a = b
b = a
\end{python}

\begin{table}[]
    \begin{tabular}{|l|l|l|l|l|}
    \hline
      & 1       & 2     & 3     & 4 
    \\ \hline
    a &    &  &   & 
    \\ \hline
    b &       &   &  & 
    \\ \hline
    \end{tabular}
\end{table}

\end{frame} 

\begin{frame}[fragile] 
	\frametitle{Tilordningsoperatoren} 
\begin{python}
a = "hei"
b = 3.14
a = b
b = a
\end{python}

\begin{table}[]
    \begin{tabular}{|l|l|l|l|l|}
    \hline
      & 1       & 2     & 3     & 4 
    \\ \hline
    a & "hei"   &  &   &  
    \\ \hline
    b & ?       &   &   & 
    \\ \hline
    \end{tabular}
\end{table}

\end{frame} 

\begin{frame}[fragile] 
	\frametitle{Tilordningsoperatoren} 
\begin{python}
a = "hei"
b = 3.14
a = b
b = a
\end{python}

\begin{table}[]
    \begin{tabular}{|l|l|l|l|l|}
    \hline
      & 1       & 2     & 3     & 4 
    \\ \hline
    a & "hei"   & "hei" &   & 
    \\ \hline
    b & ?       & 3.14  &   & 
    \\ \hline
    \end{tabular}
\end{table}

\end{frame} 

\begin{frame}[fragile] 
	\frametitle{Tilordningsoperatoren} 
\begin{python}
a = "hei"
b = 3.14
a = b
b = a
\end{python}

\begin{table}[]
    \begin{tabular}{|l|l|l|l|l|}
    \hline
      & 1       & 2     & 3     & 4 
    \\ \hline
    a & "hei"   & "hei" & 3.14  & 
    \\ \hline
    b & ?       & 3.14  & 3.14  & 
    \\ \hline
    \end{tabular}
\end{table}

\end{frame} 

\begin{frame}[fragile] 
    \frametitle{Tilordningsoperatoren} 
    
    Problem: Gammel verdi blir overskrevet. Vi kan bare endre en ting av gangen. 

\begin{python}
a = "hei"
b = 3.14
a = b
b = a
\end{python}

\begin{table}[]
    \begin{tabular}{|l|l|l|l|l|}
    \hline
      & 1       & 2     & 3     & 4 
    \\ \hline
    a & "hei"   & "hei" & 3.14  & 3.14 
    \\ \hline
    b & ?       & 3.14  & 3.14  & 3.14 
    \\ \hline
    \end{tabular}
\end{table}

\end{frame} 

\begin{frame}[fragile]
    \frametitle{Tilordningsoperatoren}

    To vanlige løsninger

\begin{python}[fragile]
a = "hei"
b = 3.14
old_a = a   # ofte kalt temporary / temp
a = b
b = old_a
\end{python}

\begin{table}[]
    \begin{tabular}{|l|l|l|l|l|l|}
    \hline
           & 1     & 2     & 3     & 4     & 5     \\ \hline
    a      & "hei" & "hei" & "hei" & 3.14  & 3.14  \\ \hline
    b      & ?     & 3.14  & 3.14  & 3.14  & "hei" \\ \hline
    a\_old & ?     & ?     & "hei" & "hei" & "hei" \\ \hline
    \end{tabular}
\end{table}

\end{frame}

\begin{frame}[fragile]
    \frametitle{Tilordningsoperatoren}

    To vanlige løsninger

\begin{python}[fragile]
a = "hei"
b = 3.14
(b, a) = (a, b) 
# bruker tupler
# kommer tilbake til dette senere
\end{python}

\end{frame}

\begin{frame}[fragile]
    \frametitle{Tilordningsoperatoren}

    Flere tilordninger etter hverandre 

\begin{python}
a = b = c = 3.14
\end{python}

\end{frame}


\begin{frame}[fragile]
    \frametitle{Tilordningsoperatoren}

    Flere tilordninger etter hverandre 

    \textit{Triks}: Parenteser. 

\begin{python}
a = (b = (c = 3.14)) 
# ikke riktig syntaks, kun for illustrere
\end{python}

\end{frame}

\begin{frame}[fragile]
    \frametitle{Flere tilordningsoperatorer}

    Finnes flere tilordningsoperatorer. 
    
    \textbf{VIKTIG}: Bortsett fra standard tilordning, må variabelen som tilordnes har en verdi fra før av. 

    \begin{table}[]
        \begin{tabular}{|l|l|l|}
        \hline
        Operator & Eksempel & Det samme som \\ \hline
        =        & a = 1    & a = 1         \\ \hline
        -=       & a -= 1   & a = a - 1     \\ \hline
        /=       & a /= 1   & a = a / 1     \\ \hline
        //=      & a //= 1  & a = a // 1    \\ \hline
        +=       & a += 1   & a = a + 1     \\ \hline
        *=       & a *= 1   & a = a * 1     \\ \hline
        \%=      & a \%= 1  & a = a \% 1    \\ \hline
        **=      & a **= 1  & a = a**1      \\ \hline
        \end{tabular}
        \end{table}

\end{frame}




\begin{frame}
    \frametitle{Aritmetiske operatorer}

    \textbf{Viktig}: Holde tunga rett i munnen med $//$, $\%$ og $**$

\begin{table}[]
    \begin{tabular}{|l|l|}
    \hline
    Operator & Eksempel \\ \hline
    +        & 1 + 2    \\ \hline
    -        & 1 - 2    \\ \hline
    /        & 1 / 2    \\ \hline
    //       & 1 // 2   \\ \hline
    *        & 1 * 2    \\ \hline
    \%       & 1 \% 2   \\ \hline
    **       & 1**2     \\ \hline
    \end{tabular}
\end{table}    

\end{frame}



\begin{frame}[fragile]
    \frametitle{Aritmetiske operatorer}

    Hva skjer her?

\begin{python}
from math import sqrt
a = sqrt(2)
b = 2**(1/2)

print(a == b)
\end{python}

\end{frame}


\begin{frame}[fragile]
    \frametitle{Aritmetiske operatorer}

    Hva skjer her?

    True printes. Hvorfor trenger man da \textit{sqrt} funksjonen? 

\begin{python}
from math import sqrt
a = sqrt(2)
b = 2**(1/2)

print(a == b)
\end{python}

\end{frame}



\begin{frame}[fragile]
    \frametitle{Sammenligningsoperatorer}

    Sammenligningsoperatorer og betingede hendelser er kanskje det viktigste med datamaskiner. Det er nødvendig for å gjøre en datamaskin \textit{Turing-complete}.

\end{frame}

\begin{frame}[fragile]
    \frametitle{Sammenligningsoperatorer}

    Boolske variabler er essensielle for sammenligningsoperatorer. En boolsk variabel kan kun være to verdier, \textit{True} eller \textit{False}. 

\end{frame}

\begin{frame}
    \frametitle{Sammenligningsoperatorer}

    Sammenligningsoperatorer og logiske operatorer. 

    \begin{table}[]
        \begin{tabular}{|l|l|l|}
        \hline
        Operator        & Eksempel                        & Resultat \\ \hline
        \textless{}     & 2 \textless 2                   & False    \\ \hline
        \textgreater{}  & 2 \textgreater 2                & False    \\ \hline
        \textless{}=    & 2 \textless{}= 2                & True     \\ \hline
        \textgreater{}= & 2 \textgreater{}= 2             & True     \\ \hline
        ==              & 2 == 2                          & True     \\ \hline
        !=              & 2 != 2                          & False    \\ \hline
        and             & True and False                  & False    \\ \hline
        or              & True or False                   & True     \\ \hline
        not             & not False                       & True     \\ \hline
        in              & 1 in {[}3, "hei", 1{]}          & True     \\ \hline
        not in          & "hade" not in {[}3, "hei", 1{]} & True     \\ \hline
        \end{tabular}
    \end{table}

\end{frame}

\begin{frame}
    \frametitle{Betingelser}

    \textit{if}, \textit{elif} og \textit{else} brukes for å kontrollere programflyten. Man kan dermed aktivere en utvalgt del av koden ved å sjekke boolske verdier. 

\end{frame}

\begin{frame}[fragile]
    \frametitle{Betingelser}

    \textit{if}, \textit{elif} og \textit{else} brukes for å kontrollere programflyten. 
    
    Man kan dermed aktivere en utvalgt del av koden ved å sjekke boolske verdier. 

\begin{python}
num = 1
if num == 0:
    print("num er 0")
elif num == 1:
    print("num er 1")
else:
    print("num er noe annet")
\end{python}

\end{frame}

\begin{frame}[fragile]
    \frametitle{Betingelser}

    \textit{if}, \textit{elif} og \textit{else} brukes for å kontrollere programflyten. 

    Man kan kjede så mange \textit{elif}-er etter hverandre. 

\begin{python}
num = 1
if num == 0:
    print("num er 0")
elif num == 1:
    print("num er 1")
elif num == 2:
    print("num er 2")
elif num == 3:
    print("num er 3")
else:
    print("num er noe annet")
\end{python}

\end{frame}

\begin{frame}[fragile]
    \frametitle{Betingelser}

    \textit{if}, \textit{elif} og \textit{else} brukes for å kontrollere programflyten. 

    Man kan kjede så mange \textit{elif}-er etter hverandre. 

    \textbf{Viktig}: Kun en av kode-snuttene vil kjøre mellom en \textit{if} og \textit{else}! 

\begin{python}
num = 1
if num == 0:
    print("num er 0")
elif num == 1:
    print("num er 1")
elif num == 2:
    print("num er 2")
elif num == 3:
    print("num er 3")
else:
    print("num er noe annet")
\end{python}

\end{frame}

\begin{frame}[fragile]
    \frametitle{Betingelser}

    \textit{if}, \textit{elif} og \textit{else} brukes for å kontrollere programflyten. 

    Man kan kjede så mange \textit{elif}-er etter hverandre. 

    \textbf{Viktig}: Kun en av kode-snuttene vil kjøre mellom en \textit{if} og \textit{else}! 

    Her er det nye \textit{if}-er, og dermed kan flere aktiveres samtidig.

\begin{python}
num = 1
if num < 0:
    print("num mindre enn 0")
if num < 1:
    print("num mindre enn 1")
if num < 2:
    print("num mindre enn 2")
if num < 3:
    print("num mindre enn 3")
else:
    print("num er noe annet")
\end{python}

\end{frame}




\begin{frame}[fragile]
    \frametitle{Datatyper}

    Variabler har bestemte typer, men kan byttes med tilordningsoperatoren. 

\begin{python}
a = "hei" # type er str
a = 1 # type er int
\end{python}

\end{frame}

\begin{frame}[fragile]
    \frametitle{Datatyper}

    Noen aritmetiske operatorer fungerer for andre typer enn tall, men ikke alltid. 

\begin{python}
liste1 = [1, 2, 3]
liste2 = [4, 5]
liste3 = liste1 + liste2

liste4 = liste3 * 2
\end{python}

\end{frame}


\begin{frame}[fragile]
    \frametitle{Datatyper}

    Eksempler på datatyper: 

    \begin{itemize}
        \item int: heltall, 1023913213, 0, -1
        \item float: flyttall, 0.1, 1.2e-10, 3.1415e6
        \item str: streng, "hei", "sacasdcasdc"
        \item list: liste, [1, 2, 3], ["hei", 1.3e2]
        \item tuple: tupler, (1, 2, 3), ("hei", 1.3e2)
    \end{itemize}

\end{frame}

\begin{frame}
    \frametitle{Datatyper}

    Funksjoner for å konvertere mellom datatyper. De gjør forskjellige ting for forskjellig input-typer, og fungerer ikke for alle. 

    \begin{table}[]
        \begin{tabular}{|l|l|l|}
        \hline
        Funksjon & Eksempel           & Resultat    \\ \hline
        bin()    & bin(92)            & '0b1011100' \\ \hline
        bool()   & bool(12), bool(0)  & True, False \\ \hline
        chr()    & chr(97)            & 'a'         \\ \hline
        ord()    & ord('a')           & 97          \\ \hline
        float()  & float(1)           & 1.0         \\ \hline
        int()    & int(2.6), int("2") & 2, 2        \\ \hline
        list()   & list(range(2))     & {[}0, 1{]}  \\ \hline
        str()    & str(100)           & '100'       \\ \hline
        \end{tabular}
    \end{table}

\end{frame}

\begin{frame}[fragile]
    \frametitle{Datatyper}

    Type gir hvilken datatype en variabel er, dermed kan vi sjekke med en if. Kan også bruke \textit{isinstance}

\begin{python}
a = 1.23
if type(a) == float:
    print("a er en float")
if isinstance(a, float):
    print("a er fortsatt en float")
\end{python}

\end{frame}

\begin{frame}
    \frametitle{Datatyper}

    Dette kan ofte være nyttig i funksjoner for å gjøre forskjellige ting for forskjellige parametere. 

\end{frame}



\begin{frame}
    \frametitle{Input og output}

    Hva er input og output?

\end{frame}

\begin{frame}
    \frametitle{Input og output}

    Hva er input og output?

    \begin{itemize}
        \item Kommunikasjon mellom bruker og kode
    \end{itemize}

\end{frame}

\begin{frame}
    \frametitle{Input og output}

    Hva er input og output?

    \begin{itemize}
        \item Kommunikasjon mellom bruker og kode
        \item Kommunikasjon mellom datamaskin og kode
    \end{itemize}

\end{frame}

\begin{frame}
    \frametitle{Input og output}

    Hva er input og output?

    \begin{itemize}
        \item Kommunikasjon mellom bruker og kode: \textit{input} and \textit{print}
        \item Kommunikasjon mellom datamaskin og kode: \textit{open}
    \end{itemize}

\end{frame}


\begin{frame}
    \frametitle{Input og output - print}

    \textit{print}-funksjonen. 

\end{frame}

\begin{frame}[fragile]
    \frametitle{Input og output - print}

    \textit{print}-funksjonen. 

\begin{python}
print("Hello World!")
\end{python}

\end{frame}

\begin{frame}[fragile]
    \frametitle{Input og output - print}

    \textit{print}-funksjonen. Skriver disse ut det samme? 

\begin{python}
print("Hello World!")
print("Hello ", "World!")
\end{python}

\end{frame}

\begin{frame}[fragile]
    \frametitle{Input og output - print}

    \textit{print}-funksjonen. Skriver disse ut det samme? 

    Default parameter: \textit{sep}

\begin{python}
print("Hello World!", sep=" ")
print("Hello ", "World!", sep=" ")
\end{python}

\end{frame}

\begin{frame}[fragile]
    \frametitle{Input og output - print}

    \textit{print}-funksjonen. Skriver disse ut det samme? 

    Default parameter: \textit{sep}

    Kan brukes til å endre hva som printes mellom argumentene i print-funksjonen. 

\begin{python}
print("Hello World!", sep=" ")
print("Hello ", "World!", sep="")
\end{python}

\end{frame}

\begin{frame}[fragile]
    \frametitle{Input og output - print}

    \textit{print}-funksjonen. Skriver disse ut det samme? 

\begin{python}
print("Hello World!")
print("Hello ")
print("World!")
\end{python}

\end{frame}

\begin{frame}[fragile]
    \frametitle{Input og output - print}

    \textit{print}-funksjonen. Skriver disse ut det samme? 

    Default parameter: \textit{end}

\begin{python}
print("Hello World!", end="\n")
print("Hello ", end="\n")
print("World!", end="\n")
\end{python}

\end{frame}

\begin{frame}[fragile]
    \frametitle{Input og output - print}

    \textit{print}-funksjonen. Skriver disse ut det samme? 

    Default parameter: \textit{end}

\begin{python}
print("Hello World!", end="\n")
print("Hello ", end="")
print("World!", end="\n")
\end{python}

\end{frame}

\begin{frame}[fragile]
    \frametitle{Input og output - print}

    \textit{sep} og \textit{end} er ikke alltid nødvendig å bruke, men kan gjøre koding lettere, og gjøre det enklere å formatere kode og plassere tekst. 

\end{frame}


\begin{frame}[fragile]
    \frametitle{Input og output - input}

    \textit{input}-funksjonen

\begin{python}
x = input("Skriv inn x")
\end{python}

\end{frame}

\begin{frame}[fragile]
    \frametitle{Input og output - input}

    \textit{input}-funksjonen.

    \textbf{Viktig}: Tar inn en (1) streng og returnerer en (1) streng! Argumenter kan ikke skrives inn som i print, vi må bruke streng-konkatenering (plusse sammen strenger). 

\begin{python}
antall_heltall = 1
x = input("Skriv inn " + str(antall_heltall) + " x: ")
x_heltall = int(x)
\end{python}

\end{frame}






\begin{frame}
    \frametitle{Funksjoner}

    Svært ofte ønsker man å gjenbruke deler av koden man skriver. 

\end{frame}


\begin{frame}
    \frametitle{Funksjoner}

    Svært ofte ønsker man å gjenbruke deler av koden man skriver. Det å printe / skrive til skjermen er egentlig en veldig kompleks prosess, men den har blitt \textit{abstrahert bort}. Abstraksjoner er veldig viktige for å løse oppgaver, både til eksamen og senere. 

\end{frame}

\begin{frame}
    \frametitle{Funksjoner}

    Et viktig prinsipp er det å \textit{kalle} en funksjon. Når man \textit{definerer} en funksjon, lager man bare en generell oppskrift for ubestemte parametere. 
    
    Når man kaller på en funksjon, så utfører man den oppskriften. Det vil si at vi gir det noen \textit{fysiske} verdier å jobbe med. 

\end{frame}


\begin{frame}
    \frametitle{Funksjoner - Funksjoner uten retur-verdi}

    I mange tilfeller skal en funksjon gjøre noe enkelt som ikke krever at koden husker på endringen etter funksjonen har blitt kjørt. 

\end{frame}

\begin{frame}
    \frametitle{Funksjoner - Funksjoner uten retur-verdi}

    I mange tilfeller skal en funksjon gjøre noe enkelt som ikke krever at koden husker på endringen etter funksjonen har blitt kjørt. 

    Dette er ofte relatert til \textit{output}, som f.eks. printing og skrive til fil. 

\end{frame}

\begin{frame}[fragile]
    \frametitle{Funksjoner - Funksjoner uten retur-verdi}

    Funksjoner begynner med kodeordet \textit{def}, funksjonsnavnet, en parentes med alle parameterene funksjonen tar inn og \textbf{VIKTIG} kolon :. Det må også være minst en kodesnutt i funksjonen.

    Kanskje den enkleste funksjonen?

\begin{python}
def func():
    pass
\end{python}

\end{frame}

\begin{frame}[fragile]
    \frametitle{Funksjoner - Funksjoner uten retur-verdi}

    Eksempel med printing. 

\begin{python}
def print_resultater(res, tekst):
    print("Resultatet ble",res,sep=": ",end=" og ")
    print("dette er teksten: " + tekst)
\end{python}

\end{frame}

\begin{frame}[fragile]
    \frametitle{Funksjoner - Funksjoner uten retur-verdi}

    Vi har allerede jobbet med default-parametere. \textit{sep} og \textit{end} er default-verdier i \textit{print}-funksjonen, og med mindre de spesifiseres vil \textit{sep} alltid være " " (mellomrom) og \textit{end} alltid være "\textbackslash n" (ny linje). 

    Her bruker vi det til å bestemme om vi skal printe eller lagre til fil. 

\begin{python}
def output_resultater(res, file_path=""):
    if file_path != "":
        f = open(file_path, "w")
        f.write(res)
        f.close()
    else:
        print("Her er resultatet:", res)

output_resultater("hei") # printes
output_resultater("hei", "test.txt") # skrives til fil
output_resultater("hei", file_path="test.txt")
\end{python}

\end{frame}


\begin{frame}
    \frametitle{Funksjoner - Funksjoner med retur-verdi}

    Dersom man ikke returnerer en verdi fra en funksjon, kan man ikke egentlig forvente at noe endres utenfor funksjonen vår. 

\end{frame}

\begin{frame}
    \frametitle{Funksjoner - Funksjoner med retur-verdi}

    Dersom man ikke returnerer en verdi fra en funksjon, kan man ikke egentlig forvente at noe endres utenfor funksjonen vår. 

    Dette er ikke helt korrekt, men kommer tilbake til dette senere. 

\end{frame}

\begin{frame}[fragile]
    \frametitle{Funksjoner - Funksjoner med retur-verdi}

    Det vi vil skal komme ut av funksjonen vår, det som skal returneres, skriver vi etter return. 

\begin{python}
def pluss(tall_1, tall_2):
    tall_sum = tall_1 + tall_2
    return tall_sum
\end{python}

\end{frame}

\begin{frame}[fragile]
    \frametitle{Funksjoner - Funksjoner med retur-verdi}

    \textbf{Viktig}: Når en verdi returneres fra en funksjon, så kan det tolkes som at vi \textit{bytter} ut funksjonskallet med 

\begin{python}
def pluss(tall_1, tall_2):
    tall_sum = tall_1 + tall_2
    return tall_sum
\end{python}

\end{frame}



\begin{frame}
    \frametitle{Iterables}

    Iterables er generelt grupper med objekter man kan \textit{iterere} seg gjennom. 

\end{frame}


\begin{frame}[fragile]
    \frametitle{Lister}

    Lister er en gruppe med objekter som ligger etter hverandre i \textit{minnet}. 
    
    De defineres med firkant-parenteser rundt objekter som er separert med komma. 
    
    De kan være av forskjellige typer, men det er ikke alltid lurt. Da kan man ikke behandle elementene likt. 

\begin{python}
liste = ["hei", 1, 0.1]
\end{python}

\end{frame}

\begin{frame}[fragile]
    \frametitle{Lister}

    Lister kan endres.

\begin{python}
liste = ["hei", 1, 0.1]
liste[1] = "alle"
\end{python}

\end{frame}

\begin{frame}[fragile]
    \frametitle{Lister}

    Lister kan utvides. 

\begin{python}
liste = ["hei", 1, 0.1]
liste = liste + ["alle", "sammen"]
\end{python}

\end{frame}

\begin{frame}
    \frametitle{Lister}

    Alle gjør endringer på den originale listen, \textbf{bortsett fra} \textit{copy()}, som returnerer en ny liste. Dette er viktig dersom man vil beholde originalen. 

    \begin{table}[]
        \begin{tabular}{|l|l|l|l|l|}
        \hline
        Metode & Eksempel               &  & Metode  & Eksempel        \\ \hline
        append & liste.append("hei")    &  & reverse & liste.reverse() \\ \hline
        insert & liste.insert(2, "hei") &  & sort    & liste.sort()    \\ \hline
        remove & liste.remove("hei")    &  & copy    & liste.copy()    \\ \hline
        pop    & liste.pop(1)           &  & clear   & kopi = liste.clear()   \\ \hline
        count  & liste.count("hei")     &  &         &                 \\ \hline
        \end{tabular}
    \end{table}

\end{frame}

\begin{frame}[fragile]
    \frametitle{Lister}

    Når man bruker lister i funksjoner, så må man huske at hvis man endrer listen man tar inn, så endres den utenfor også! 
    
    Dette gjelder ikke for vanlige variabler/parametere.

\begin{python}
def sett_tall(tall):
    tall = 1
def legg_til(liste):
    liste.append(1)
a = "hei"
sett_tall(a) # a endres ikke
l = [1, 5.12]
legg_til(l) # l endres
\end{python}

\end{frame}


\begin{frame}
    \frametitle{Tupler}

    Tupler er basically det samme som lister, men de er ikke-muterbare, som strenger. 

\end{frame}

\begin{frame}[fragile]
    \frametitle{Tupler}

    Defineres på samme måte som lister, bare med parenteser istedenfor firkantparenteser.

\begin{python}
liste = [1, 2, 3]
tuple = (1, 2, 3)
\end{python}

\end{frame}

\begin{frame}[fragile]
    \frametitle{Tupler}

    Metoder: Siden tupler er ikke-muterbare, er det svært få metoder, og de er generelt ikke nødvendige å kunne. 

    \begin{itemize}
        \item count
        \item index
    \end{itemize}

\begin{python}
t = (1, 2, 3, 1)
t.count(1) # -> 2
t.index(1) # -> 0
\end{python}

\end{frame}

\begin{frame}
    \frametitle{Tupler}

    Hvorfor bruke tupler? 

\end{frame}


\begin{frame}[fragile]
    \frametitle{Tupler}

    Hvorfor bruke tupler? 

    Tupler kan brukes der du vet du har en mengde som er konstant

\begin{python}
mynter = (20, 10, 5, 1)
penger = 19
liste  = [0] * len(mynter)
for i in range(len(mynter)): 
    liste[i] = penger \\ mynter[i]
    penger  %= mynter[i]
\end{python}

\end{frame}

\begin{frame}[fragile]
    \frametitle{Tupler}

    Kan forenkle kode som trenger flere linjer. Tenk tilbake til tilordningsproblemet der vi ville bytte om på verdiene i to variabler.

\begin{python}
a = "hei"
b = 2
temp = a
a = b
b = temp
\end{python}

\end{frame}

\begin{frame}[fragile]
    \frametitle{Tupler}

    Kan forenkle kode som trenger flere linjer. Tenk tilbake til tilordningsproblemet der vi ville bytte om på verdiene i to variabler. Dette kan nå forenkles. 

\begin{python}
a = "hei"
b = 2
a, b = b, a
\end{python}

\end{frame}

\begin{frame}[fragile]
    \frametitle{Tupler}

    Mange nyttige funksjoner man kan få bruk for bruker også tupler. 

\begin{python}
for index, element in enumerate(liste)
for key, item in dictionary.items()
\end{python}

\end{frame}

\begin{frame}
    \frametitle{Tupler}

    Generelt: hvis du kan gjøre det med tupler, kan det gjøres med lister eller med annen kode. Tupler er hovedsakelig \textit{ekstra}. 

\end{frame}







\begin{frame}
    \frametitle{Løkker}

    Løkker er helt essensielle i koding. 

\end{frame}

\begin{frame}
    \frametitle{Løkker}

    Løkker er helt essensielle i koding. 

    Idéen bak er å repetere en viss kodesnutt om og om igjen.

\end{frame}



\begin{frame}
    \frametitle{Løkker - while-løkken}

    Strengt tatt er alle løkker en while løkke. 

\end{frame}

\begin{frame}[fragile]
    \frametitle{Løkker - while-løkken}

    While-løkker kjører kun hvis en boolsk verdi er True. Det betyr at vi kan sjekke en condition i løkka. 

\begin{python}
while True:
    print("to infinity!")
print("and beyond?")
\end{python}

\end{frame}

\begin{frame}[fragile]
    \frametitle{Løkker - while-løkken}

    Minner om en for-løkke? 

\begin{python}
i = 0
while i < 10:
    print(i)
    i += 1
\end{python}

\end{frame}

\begin{frame}[fragile]
    \frametitle{Løkker - while-løkken}

    Veldig bra å bruke dersom man ikke vet hvor mange ganger løkka skal kjøre, men \textit{vet det når man ser det}. 

\begin{python}
# dette er en metode som sikrer at det tomme svaret 
# ikke blir tatt med
answer = input("Skriv inn ditt svar: ")
answer_lst = []
while answer != "": # tomt svar, trykket enter med en gang 
    answer_lst.append(answer) 
    answer = input("Skriv inn ditt svar: ")
\end{python}

\end{frame}


\begin{frame}
    \frametitle{Løkker - for-løkken}

    for-løkker brukes når man har en bestemt \textit{iterable} man ønsker å iterere gjennom. 

\end{frame}


\begin{frame}[fragile]
    \frametitle{Løkker - for-løkken}

    for-løkker brukes når man har en bestemt \textit{iterable} man ønsker å iterere gjennom. Dette gjelder spesielt de vi allerede har gått gjennom, i form av \textit{lister} og \textit{tupler}. 

\end{frame}

\begin{frame}[fragile]
    \frametitle{Løkker - for-løkken}

    for-løkker brukes når man har en bestemt \textit{iterable} man ønsker å iterere gjennom. Dette gjelder spesielt de vi allerede har gått gjennom, i form av \textit{lister} og \textit{tupler}. 

    Disse kan defineres \textit{inline}

\begin{python}
for word in ("hei", "hei", "alle", "sammen"): 
    print(word)
for word in ["hei", "hei", "alle", "sammen"]: 
    print(word)
\end{python}

\end{frame}

\begin{frame}[fragile]
    \frametitle{Løkker - for-løkken}

    for-løkker brukes når man har en bestemt \textit{iterable} man ønsker å iterere gjennom. Dette gjelder spesielt de vi allerede har gått gjennom, i form av \textit{lister} og \textit{tupler}. 

    Eller på forhånd. Da kan de gjenbrukes senere, eller redefineres før. 

\begin{python}
tpl = ("hei", "hei", "alle", "sammen")
lst = ["hei", "hei", "alle", "sammen"]
for word in tpl: 
    print(word)
for word in lst: 
    print(word)
\end{python}

\end{frame}

\begin{frame}
    \frametitle{Løkker - for-løkken}

    Det er hovedsakelig to måter å \textit{iterere} gjennom en liste. 

\end{frame}

\begin{frame}[fragile]
    \frametitle{Løkker - for-løkken}

    Det er hovedsakelig to måter å \textit{iterere} gjennom en liste. 

    Element-vis. Man henter ut hvert \textit{element} i listen.

\begin{python}
lst = [1, 2, 3, "hei", (1, 2, 3)]
for element in lst: 
    print(element)
\end{python}

\end{frame}

\begin{frame}[fragile]
    \frametitle{Løkker - for-løkken}

    Det er hovedsakelig to måter å \textit{iterere} gjennom en liste. 

    Indeks-vis. Man henter ut \textit{plasseringen} til hvert element i listen. 

\begin{python}
lst = [1, 2, 3, "hei", (1, 2, 3)]
for i in range(len(lst)): 
    print(lst[i])
\end{python}

\end{frame}

\begin{frame}[fragile]
    \frametitle{Løkker - for-løkken}

    Det er hovedsakelig to måter å \textit{iterere} gjennom en liste. 

    Indeks-vis. Man henter ut \textit{plasseringen} til hvert element i listen. 

    Viktig: hvis vi itererer element-vis gjennom en liste, vet vi egentlig ingenting om plasseringen, og hvis vi trenger indeksen, er det best å gå gjennom indeks-vis. 

\begin{python}
lst = [1, 2, 3, "hei", (1, 2, 3)]
for i in range(len(lst)): 
    print(lst[i])
\end{python}

\end{frame}

\begin{frame}[fragile]
    \frametitle{Løkker - for-løkken}

    Det er hovedsakelig to måter å \textit{iterere} gjennom en liste. 

    Kan bruke \textit{enumerate} for å få ut begge med en gang. 

\begin{python}
lst = [1, 2, 3, "hei", (1, 2, 3)]
for i, element in enumerate(lst): 
    # lst[i] == element -> True
    print(i, lst[i], element)
\end{python}

\end{frame}

\begin{frame}[fragile]
    \frametitle{Løkker - for-løkken}

    Det er hovedsakelig to måter å \textit{iterere} gjennom en liste. 

    Kan bruke \textit{enumerate} for å få ut begge med en gang. Sjeldent dette ikke kan løses med bare indekser, og tilordning. 

\begin{python}
lst = [1, 2, 3, "hei", (1, 2, 3)]

for i, element in enumerate(lst): 
    # lst[i] == element -> True
    print(i, lst[i], element)

for i in range(len(lst)):
    element = lst[i]
    # lst[i] == element -> True
    print(i, lst[i], element)
\end{python}

\end{frame}

\begin{frame}[fragile]
    \frametitle{Løkker - for-løkken}

    Løkker kan enkelt skrives inne i hverandre. 

\begin{python}
def antall_like(lst1, lst2):
    res = 0
    for x in lst1: 
        for y in lst2:
            if x == y: 
                res += 1
    
    return res
\end{python}

\end{frame}

\begin{frame}[fragile]
    \frametitle{Løkker - for-løkken}

    Løkker kan enkelt skrives inne i hverandre. Går gjennom hvert element i \textit{lst1}, sjekker hvert element i \textit{lst2} og ser om jeg finner det. 

\begin{python}
def antall_like(lst1, lst2):
    res = 0
    for x in lst1: 
        for y in lst2:
            if x == y: 
                res += 1
    
    return res
\end{python}

\end{frame}

\begin{frame}[fragile]
    \frametitle{Løkker - for-løkken}

    Løkker kan enkelt skrives inne i hverandre. Går gjennom hvert element i \textit{lst1}, sjekker hvert element i \textit{lst2} og ser om jeg finner det. 

    Her trenger vi ikke gå gjennom med indekser (range), siden vi trenger ikke vite noe om posisjon. 

\begin{python}
def antall_like(lst1, lst2):
    res = 0
    for x in lst1: 
        for y in lst2:
            if x == y: 
                res += 1
    
    return res
\end{python}

\end{frame}

\begin{frame}[fragile]
    \frametitle{Løkker - for-løkken}

    Det kan se vanskelig ut å analysere når det er flere løkker. 

\begin{python}
def antall_like(lst1, lst2):
    res = 0
    for x in lst1: 
        for y in lst2:
            if x == y: 
                res += 1
    
    return res
\end{python}

\end{frame}

\begin{frame}[fragile]
    \frametitle{Løkker - for-løkken}

    Det kan se vanskelig ut å analysere når det er flere løkker. Det er enklest å begynne å analysere \textit{innenfra og ut}.

\begin{python}
def antall_like(lst1, lst2):
    res = 0
    for x in lst1: 
        for y in lst2:
            if x == y: 
                res += 1
    
    return res
\end{python}

\end{frame}

\begin{frame}[fragile]
    \frametitle{Løkker - for-løkken}

    Det kan se vanskelig ut å analysere når det er flere løkker. Det er enklest å begynne å analysere \textit{innenfra og ut}. Dette gjelder også når vi skal designe kode som bruker kode inni kode. 

\begin{python}
def antall_like(lst1, lst2):
    res = 0
    for x in lst1: 
        for y in lst2:
            if x == y: 
                res += 1
    
    return res
\end{python}

\end{frame}

\begin{frame}[fragile]
    \frametitle{Løkker - for-løkken}

    Vi antar at \textit{x} og \textit{y} er bestemte verdier (f.eks. \textit{x} = 1, \textit{y} = 2). Hva er det vi vil gjøre med denne bestemte verdien? 

\begin{python}
def antall_like(lst1, lst2):
    res = 0
    for x in lst1: 
        for y in lst2:
            if x == y: 
                res += 1
    
    return res
\end{python}

\end{frame}

\begin{frame}[fragile]
    \frametitle{Løkker - for-løkken}

    Vi antar at \textit{x} og \textit{y} er bestemte verdier (f.eks. \textit{x} = 1, \textit{y} = 2). Hva er det vi vil gjøre med denne bestemte verdien? 

    Hvis disse er like, øker vi \textit{telleren} med 1. 

\begin{python}
def antall_like(lst1, lst2):
    res = 0
    for x in lst1: 
        for y in lst2:
            if x == y: 
                res += 1
    
    return res
\end{python}

\end{frame}

\begin{frame}[fragile]
    \frametitle{Løkker - for-løkken}

    Vi antar at \textit{x} og \textit{y} er bestemte verdier (f.eks. \textit{x} = 1, \textit{y} = 2). Hva er det vi vil gjøre med denne bestemte verdien? 

    Hvis disse er like, øker vi \textit{telleren} med 1. 

    Nå er den \textit{innerste} delen av koden abstrahert bort. Hva er neste steg? 

\begin{python}
def antall_like(lst1, lst2):
    res = 0
    for x in lst1: 
        for y in lst2:
            if x == y: 
                res += 1
    
    return res
\end{python}

\end{frame}

\begin{frame}[fragile]
    \frametitle{Løkker - for-løkken}

    Vi antar at \textit{x} er en bestemt verdi (f.eks. \textit{x} = 1). 

    Vi vet hva som skjer for en bestemt \textit{y}, så det kan generaliseres. Vi vet hva \textit{x} er, og vi vet hva vi skal gjøre hvis vi har en \textit{x} og en \textit{y}. Så vi går gjennom alle mulige \textit{y}-er. 

\begin{python}
def antall_like(lst1, lst2):
    res = 0
    for x in lst1: 
        for y in lst2:
            if x == y: 
                res += 1
    
    return res
\end{python}

\end{frame}

\begin{frame}[fragile]
    \frametitle{Løkker - for-løkken}

    Nå trenger vi ikke lenger å se på en bestemt verdi for \textit{x}, siden vi vet hva som skjer med en (den sammenlignes med hver mulige \textit{y}), kan den også generaliseres. 

    For hver mulige \textit{x}, sjekk hver mulige \textit{y}. Dersom \textit{x} == \textit{y}, så øker vi telleren med en.

\begin{python}
def antall_like(lst1, lst2):
    res = 0
    for x in lst1: 
        for y in lst2:
            if x == y: 
                res += 1
    
    return res
\end{python}

\end{frame}

\begin{frame}[fragile]
    \frametitle{Løkker - for-løkken}

    Strenger og tupler kan behandles ganske likt, siden begge er ikke-muterbare. F.eks., dersom vi vil endre noe, så må vi bygge disse strukturene på nytt. 

    Dette kan vi gjøre litt enklere senere med slicing. 

\begin{python}
def insert_string(original, index, new_string):
    # antar at index er valid
    resultat = "" # starter med tom streng 
    for i in range(len(original)): 
        if i == index: 
            resultat += new_string 
        # om vi plusser foran eller bak new_string 
        # bestemmer om vi inserter foran eller bak. 
        resultat += original[i] 
    return resultat
\end{python}

\end{frame}
















\begin{frame}
    \frametitle{Iterables}

    Iterables er generelt grupper med objekter man kan \textit{iterere} seg gjennom. 

\end{frame}


\begin{frame}
    \frametitle{Dictionaries}

    Dictionaries er kanskje den nyttigeste datastrukturen i Python, og en av de mest brukte. 

\end{frame}

\begin{frame}
    \frametitle{Dictionaries}

    Dictionaries er kanskje den nyttigeste datastrukturen i Python, og en av de mest brukte. Det som er viktig å huske på er at en dictionary \textit{kobler} sammen ulike verdier. 

\end{frame}

\begin{frame}
    \frametitle{Dictionaries}

    Dictionaries er kanskje den nyttigeste datastrukturen i Python, og en av de mest brukte. Det som er viktig å huske på er at en dictionary \textit{kobler} sammen ulike verdier. 

    Vi kobler fra en \textit{nøkkel} til en \textit{verdi}.

\end{frame}

\begin{frame}[fragile]
    \frametitle{Dictionaries}

    Syntaksen fungerer i prinsipp på samme måte som en liste. Vi bruker firkantparenteser til å aksessere \textit{verdien} som ligger på en \textit{nøkkel}.

\begin{python}
dictionary = {"katt": 1, "hund": 2}
print(dictionary["katt"]) # gir 1
\end{python}

\end{frame}

\begin{frame}[fragile]
    \frametitle{Dictionaries}

    Kan forenkle problemer som kan være vanskelige med lister. Eksempel: Telle opp antall bokstaver i en streng. 

\begin{python}
streng = "heihei"
# alle bokstaver fra a til z
bokstaver = [chr(bokstav) for bokstav in range(ord("a"), ord("z") + 1, 1)]
oppteller = [0] * len(bokstaver)
for i in range(len(bokstaver)): 
    oppteller[i] = streng.count(bokstaver[i])
\end{python}

\end{frame}

\begin{frame}[fragile]
    \frametitle{Dictionaries}

    Kan forenkle problemer som kan være vanskelige med lister. Eksempel: Telle opp antall bokstaver i en streng. 

    Kan være vanskelig å håndtere dette, siden vi har to separate lister. Finnes selvsagt andre løsninger, men for å få en direkte kobling mellom bokstav og antall er det enklest med dictionary. 

\begin{python}
streng = "heihei"
bokstaver_til_antall = {}
for bokstav in streng: 
    antall = bokstaver_til_antall.setdefault(bokstav, 0)
    bokstaver_til_antall[bokstav] = antall + 1
\end{python}

\end{frame}


\begin{frame}
    \frametitle{Sets}

    Det finnes også set, som ligner veldig på lister. Disse kan kun ha unike verdier, som er egentlig eneste grunn til å bruke disse. Problemer man kan løse med set kan også løses med lister, men man kan få kortere (men muligens mindre leslige) løsninger. 

\end{frame}



\begin{frame}
    \frametitle{Input og output}

    Hva er input og output?

\end{frame}

\begin{frame}
    \frametitle{Input og output}

    Hva er input og output?

    \begin{itemize}
        \item Kommunikasjon mellom bruker og kode
    \end{itemize}

\end{frame}

\begin{frame}
    \frametitle{Input og output}

    Hva er input og output?

    \begin{itemize}
        \item Kommunikasjon mellom bruker og kode
        \item Kommunikasjon mellom datamaskin og kode
    \end{itemize}

\end{frame}

\begin{frame}
    \frametitle{Input og output}

    Hva er input og output?

    \begin{itemize}
        \item Kommunikasjon mellom bruker og kode: \textit{input} and \textit{print}
        \item Kommunikasjon mellom datamaskin og kode: \textit{open}
    \end{itemize}

\end{frame}


\begin{frame}
    \frametitle{Input og output - Lese filer}

    \textit{open}-funksjonen brukes til å åpne filer generelt. Dette gjelder både lesing av og skriving til filer. 

\end{frame}


\begin{frame}
    \frametitle{Input og output - Lese filer}

    \textit{open}-funksjonen brukes til å åpne filer generelt. Dette gjelder både lesing av og skriving til filer. 

    Vi begynner med å lese fra filer som allerede eksisterer. 

\end{frame}


\begin{frame}
    \frametitle{Input og output - Lese filer}

    \textit{open}-funksjonen tar inn to parametere. Begge er strenger, den første er hvilken fil du vil åpne, og den andre er hvilken \textit{modus} du vil åpne filen i. For å lese fra filer bruker vi "r" for å lese. 
    
    Vi kan videre bestemme om vi vil åpne fila som en tekstfil eller binærfil. 

    Det vil si, "rt" og "rb", men "r" tolkes som "rt", siden det er vanligst, og med mindre det står eksplisitt på eksamen at man skal lese fila som en binærfil, er det "r" som mest sannsynlig skal brukes for å lese filer. 

\end{frame}

\begin{frame}[fragile]
    \frametitle{Input og output - Lese filer}

    Filen "test.txt" må eksistere, ellers får vi en feilmelding. I tillegg må den ligge i \textbf{samme mappe som du kjører Python-fila fra}. 

\begin{python}
f = open("test.txt", "r")
print(f.read())
f.close()
\end{python}

\end{frame}

\begin{frame}[fragile]
    \frametitle{Input og output - Lese filer}

    Filen "test.txt" må eksistere, ellers får vi en feilmelding. I tillegg må den ligge i \textbf{samme mappe som du kjører Python-fila fra}. 

    Dette kalles relativ filplassering. 

\begin{python}
f = open("test.txt", "r")
print(f.read())
f.close()
\end{python}

\end{frame}

\begin{frame}[fragile]
    \frametitle{Input og output - Lese filer}

    Metoden \textit{read()} henter ut \textbf{all} teksten fra fila. 

\begin{python}
f = open("test.txt", "r")
print(f.read())
f.close()
\end{python}

\end{frame}

\begin{frame}[fragile]
    \frametitle{Input og output - Lese filer}

    Metoden \textit{read()} henter ut \textbf{all} teksten fra fila. Det vil si at vi ikke kan lese to ganger fra samme objekt, og eventuelt må åpne fila på nytt. 

\begin{python}
f = open("test.txt", "r")
print(f.read())
f.close()
\end{python}

\end{frame}

\begin{frame}[fragile]
    \frametitle{Input og output - Lese filer}

    Metoden \textit{read()} henter ut \textbf{all} teksten fra fila. Det vil si at vi ikke kan lese to ganger fra samme objekt, og eventuelt må åpne fila på nytt. 

    Husk å lukke fila til slutt. Dette er ikke like viktig når man leser fra en fil som når man skriver til en fil, men er lurt siden man kan miste poeng på eksamen. 

\begin{python}
f = open("test.txt", "r")
print(f.read())
f.close()
\end{python}

\end{frame}

\begin{frame}[fragile]
    \frametitle{Input og output - Lese filer}

    Det finnes andre metoder også, for eksempel, dersom tekstfila er for stor til å åpnes i minnet. 

\end{frame}

\begin{frame}[fragile]
    \frametitle{Input og output - Lese filer}

    Det finnes andre metoder også, for eksempel, dersom tekstfila er for stor til å åpnes i minnet. 
    
    Metoden \textit{readline()} leser en og en linje, og når alt er lest ut returnerer den bare tomme strenger. 

\begin{python}
f = open("test.txt", "r")

f_lesing = f.readline()
print(f_lesing)
while f_lesing != "":
    f_lesing = f.readline()
    print(f_lesing)

\end{python}

\end{frame}

\begin{frame}[fragile]
    \frametitle{Input og output - Lese filer}

    Det finnes andre metoder også, for eksempel, dersom tekstfila er for stor til å åpnes i minnet. 

    Metoden \textit{readlines()} er veldig lik, men leser ut alle linjene i fila, og legger dem i en liste. 
    
    Begge disse kan være nyttige (spesielt \textit{readlines()}) dersom man skal gjøre noe pr. linje, og kan gjøre det mer intuitivt å jobbe med filer. 

\begin{python}
f = open("test.txt", "r")

f_linjer = f.readlines()
for linje in f_linjer:
    print(linje)
\end{python}

\end{frame}

\begin{frame}[fragile]
    \frametitle{Input og output - Lese filer}

    Det finnes andre metoder også, for eksempel, dersom tekstfila er for stor til å åpnes i minnet. 
    
    Begge disse kan være nyttige (spesielt \textit{readlines()}) dersom man skal gjøre noe pr. linje, og kan gjøre det mer intuitivt å jobbe med filer. 

\begin{python}
f = open("test1.txt", "r")
g = open("test2.txt", "r")

f_lesing = f.readline()
print(f_lesing)
while f_lesing != "":
    f_lesing = f.readline()
    print(f_lesing)

g_linjer = g.readlines()
for linje in g_linjer:
    print(linje)
\end{python}

\end{frame}


\begin{frame}
    \frametitle{Input og output - Skrive til filer}

    Igjen brukes \textit{open}-funksjonen. Viktig å holde tunga rett i munnen, siden det er mye som kan gå galt her.

\end{frame}


\begin{frame}[fragile]
    \frametitle{Input og output - Skrive til filer}

    Igjen brukes \textit{open}-funksjonen. Viktig å holde tunga rett i munnen, siden det er mye som kan gå galt her.

    Her er det tre nye moduser, ikke bare \textit{"r"}, som tidligere. 

    \begin{itemize}
        \item "w": Write, standard skriving. Dersom fila eksisterer fra før av, slettes den opprinnelige fila. 
        \item "a": Append, legger til tekst på slutten av fila. Dersom den ikke eksisterer, opprettes den. 
        \item "x": Exclusive write, vil feile dersom fila allerede eksisterer. Kan bare brukes til å opprette nye filer. 
    \end{itemize}

\end{frame}

\begin{frame}[fragile]
    \frametitle{Input og output - Skrive til filer}

    Igjen brukes \textit{open}-funksjonen. Viktig å holde tunga rett i munnen, siden det er mye som kan gå galt her.

    Her er det tre nye moduser, ikke bare \textit{"r"}, som tidligere. 

    \begin{itemize}
        \item "w": Write, standard skriving. Dersom fila eksisterer fra før av, slettes den opprinnelige fila. 
        \item "a": Append, legger til tekst på slutten av fila. Dersom den ikke eksisterer, opprettes den. 
        \item "x": Exclusive write, vil feile dersom fila allerede eksisterer. Kan bare brukes til å opprette nye filer. 
    \end{itemize}

    Det finnes også en modus "+", som tillater oppdatering (read og write samtidig), men er oftest bedre å gjøre dette i to steg. 

\end{frame}

\begin{frame}[fragile]
    \frametitle{Input og output - Skrive til filer}

    Igjen brukes \textit{open}-funksjonen. Viktig å holde tunga rett i munnen, siden det er mye som kan gå galt her.

    Tilsvarende som med å lese fra filer, kan filer også åpnes som binærfiler. 

\end{frame}

\begin{frame}[fragile]
    \frametitle{Input og output - Skrive til filer}

    Igjen brukes \textit{open}-funksjonen. Viktig å holde tunga rett i munnen, siden det er mye som kan gå galt her.

    Tilsvarende som med å lese fra filer, kan filer også åpnes som binærfiler. 

\end{frame}

\begin{frame}[fragile]
    \frametitle{Input og output - Skrive til filer}

    Skriving til fil er dermed ganske rett frem. Man skriver fra \textit{der man er} i fila. 

\begin{python}
f = open("test.txt", "w")
f.write("Hello\n")
f.write("World!\n")
f.close()

g = open("test.txt", "a")
g.write("New\n")
g.write("lines!\n")
g.close()
\end{python}

\end{frame}





\begin{frame}
    \frametitle{Unntak}

    Kode kjører ikke alltid som vi forventer, spesielt hvis vi har bruker-input. 

\end{frame}

\begin{frame}
    \frametitle{Unntak}

    Kode kjører ikke alltid som vi forventer, spesielt hvis vi har bruker-input. 

    Det som ofte gjøres da er at man skriver kode, som forventer at svarene lar koden kjøre, og har unntak dersom noe uventet skjer. 

\end{frame}

\begin{frame}
    \frametitle{Unntak}

    Kode kjører ikke alltid som vi forventer, spesielt hvis vi har bruker-input. 

    Det som ofte gjøres da er at man skriver kode, som forventer at svarene lar koden kjøre, og har unntak dersom noe uventet skjer. 

    Dette er en svært vanlig form for feilhåndtering. 

\end{frame}

\begin{frame}
    \frametitle{Unntak}

    Kode kjører ikke alltid som vi forventer, spesielt hvis vi har bruker-input. 

    Det som ofte gjøres da er at man skriver kode, som forventer at svarene lar koden kjøre, og har unntak dersom noe uventet skjer. 

    Dette er en svært vanlig form for feilhåndtering. 

\end{frame}

\begin{frame}[fragile]
    \frametitle{Unntak}

    Eksempel

\begin{python}
try:
    number = int(input("Write a number: "))
except ValueError as error: 
    print(error)
    print("Not a number")
except: 
    print("Some other mistake")
else:
    print("The number is:", number)
finally:
    print("Done")
\end{python}

\end{frame}

\begin{frame}[fragile]
    \frametitle{Unntak}

    Eksempel

\begin{python}
try:
    number = int(input("Write a number: ")) # runs always, but may crash
except ValueError as error: # specifics assigned to variable error
    # runs specifically if there is a ValueError
    print(error) # prints the specifics of this error
    print("Not a number")
except: # runs for any other exception
    print("Some other mistake")
else: # runs if no crash
    print("The number is:", number) 
finally: # runs always
    print("Done")
\end{python}

\end{frame}








\begin{frame}
    \frametitle{Eksamensoppgaver}

    Går gjennom noen eksamensoppgaver, både kodeforståelse og koding. 

\end{frame}

\begin{frame}
    \frametitle{Eksamensoppgaver}

    Går gjennom noen eksamensoppgaver, både kodeforståelse og koding. 
    
    Fokus på forståelse, visualisering og "tankeprosessen". 

\end{frame}



\begin{frame}
    \frametitle{Kodeforståelse}

    Viktig å lese oppgaveteksten. 
    
    Ofte er det spørsmål om både hva som returneres for en bestemt input-verdi, i tillegg til å beskrive funksjonen med en setning. 

\end{frame}


\begin{frame}
    \frametitle{Kodeforståelse}

    Viktig å lese oppgaveteksten. 
    
    Ofte er det spørsmål om både hva som returneres for en bestemt input-verdi, i tillegg til å beskrive funksjonen med en setning. 

    Det vil \textit{ikke} si at vi skriver hva som skjer, men heller hva som er poenget med funksjonen. 

\end{frame}

\begin{frame}[fragile]
    \frametitle{Kodeforståelse}

    \textbf{Feil svar:} Funksjonen setter \textit{r} = 1, og så lenge \textit{x} er større eller lik 1 ganges \textit{r} med \textit{x}, og \textit{x} minker med 1. 

\begin{python}
def f(x): 
    r = 1
    while x >= 1: 
        r *= x
        x -= 1
    return r
\end{python}

\end{frame}

\begin{frame}[fragile]
    \frametitle{Kodeforståelse}

    \textbf{Riktig(ere) svar:} Funksjonen beregner verdien $x!$, x fakultet.

\begin{python}
def f(x): 
    r = 1
    while x >= 1: 
        r *= x
        x -= 1
    return r
\end{python}

\end{frame}

\begin{frame}[fragile]
    \frametitle{Kodeforståelse 2016 2d}

    Hva blir skrevet ut på skjermen hvis koden vist under blir kjørt? 

    Forklar med en setning hva funksjonen $f$ gjør

\begin{python}
def f(x):
    y = 0
    while x > 0:
        y = y + x % 10
        x = int( x / 10 )
    if y >= 10:
        y = f( y )
    return y

print( f(32145) )
\end{python}

\end{frame}

\begin{frame}
    \frametitle{Kodeforståelse 2016 kont 4b}

    6

    $f$ regner ut den rekursive tverrsummen tallet som den får inn som parameter, dvs. fortsetter å ta tverrsum helt til vi får et ensifret tall. 

    Tverrsummen av 32145 blir 15, tverrsummen av 15 blir deretter 6. 

\end{frame}

\begin{frame}[fragile]
    \frametitle{Kodeforståelse 2016 kont 4b}

    Hva returneres hvis man kaller $f([[3, 5], [2, 4], [1, 3]])$ med koden nedenfor? 

    Forklar med \underline{en setning} hva funksjonen $f()$ gjør? 

\begin{python}
def f(b): 
    c=len(b[0])
    d=len(b)
    g = [[0 for row in range(d)]
        for col in range(c)]
    for e in range(0, c):
        for f in range(0, d): 
            g[e][f] = b[f][e]
    return g
\end{python}

\end{frame}

\begin{frame}
    \frametitle{Kodeforståelse 2016 kont 4b}

    [[3, 2, 1], [5, 4, 3]]

    Evt. ingenting ettersom det er feil med innrykk i oppgaveteksten. 
    Begge svar gir full pott!

    Funksjonen transponerer en matrise. 

\end{frame}

\begin{frame}[fragile]
    \frametitle{Kodeforståelse 2017 2a}

    Funksjonen $bin_search$ er ment til å skulle utføre binærsøk, men resulterer i feilmeldingen \textit{"IndexError: list index out of range"}. 

    I hvilken linje er feilen? 

    Hva skulle egentlig stått på den linjen for at funksjonen skal virke etter sin hensikt? 

\begin{python}
def bin_search(liste, verdi, imin, imax): 
    if (imax < imin): 
        return False
    else: 
        imid = (imin + imax)
        if verdi < liste[imid]: 
            return bin_search(liste, verdi, imin, imid-1)
        elif verdi > liste[imid]: 
            return bin_search(liste, verdi, imid+1, imax)
        else: 
            return imid
\end{python}

\end{frame}

\begin{frame}
    \frametitle{Kodeforståelse 2017 2a}

    Linje 5

    $imid = (imid + imax) / 2$

\end{frame}

\begin{frame}[fragile]
    \frametitle{Kodeforståelse 2018 sett 1 2a}

    Hva returneres ved funksjonskallet under? 
    $myst(((True and False) or (False and True)), ((False or True) and (not(not True))))$

\begin{python}
def myst(val1, val2):
    if (val1 and val2):
        return 1 
    elif (val1 and not val2): 
        return 2
    elif (not val1 and val2): 
        return 3 
    else: 
        return 4
\end{python}

\end{frame}

\begin{frame}
    \frametitle{Kodeforståelse 2017 2a}

    3

\end{frame}

\begin{frame}[fragile]
    \frametitle{Kodeforståelse 2018 kont 2e og f}

    Hva blir skrevet ut til skjerm når du kjører programmet vist under?

    $z = ((2, 0, 11, 8, 5), (14, 17, 13, 8, 0))$
    $print(myst3(z))$

    Beskriv med en setning hva koden i oppgave 2e gjør. 
    
\begin{python}
def myst3(a): 
	s  = ''
	for r in a: 
		for c in r: 
			s += chr(ord('A') + c)
	return s
\end{python}

\end{frame}

\begin{frame}
    \frametitle{Kodeforståelse 2018 kont 2e og f}

    \textbf{CALIFORNIA}
    Store bokstaver (pga. $ord('A')$) og ingen fnutter (pga. printing)

    Henter ut ett og ett tall fra et tuppel av tupler, og lager en sammenhengende tekst av store bokstaver hvor c angir hvor langt unna hver bokstav er fra A i alfabetet (ASCII- tabellen).

\end{frame}


\begin{frame}
    \frametitle{Programmering 2016 kont 2}

    Du kan anta at alle funksjonene mottar gyldige argumenter (inn-verdier). Du kan benytte deg av funksjoner fra deloppgaver selv om du ikke har løst deloppgaven. 

    I denne oppgaven skal man lage funksjoner for å lese en tekstfil med binærkode og gjøre om dette til tegn kodet i eget kodesett for tegn og bokstaver og lagre det til en tekstfil.

\end{frame}


\begin{frame}
    \frametitle{Programmering 2016 kont 2a}

    Lag funksjonen $load_bin$ som har en inn-parameter $filename$, som er navnet på fila som skal lastes inn. Funksjonen skal lese inn alt innholdet i fila og returnere innholdet som en tekststreng uten linjeskift eller mellomrom. Fila som det leses fra er en tekstfil bestående av binære tall (0 og 1). Hvis fila ikke eksisterer eller ikke kan åpnes, skal funksjonen returnere en tom streng samt skrive ut følgende feilmelding til skjerm: "Error: Could not open file <filename>", der <filename> er navnet på fila. 

\end{frame}

\begin{frame}[fragile]
    \frametitle{Programmering 2016 kont 2a}

\begin{python}
def load_bin(filename):
    binstring = ""
    try:
        f = open(filename, "r")

        for line in f: 
            binstring += line.strip()
        
        f.close()
    except:
        print("Error: Could not open file", filename)
    return binstring
\end{python}

\end{frame}

\begin{frame}
    \frametitle{Programmering 2016 kont 2b}

    Lag funksjonen $bin_to_dec$ som har en inn-parameter $binary$, som er en tekststreng av ukjent størrelse bestående av binære tall (tekststreng med nuller og enere). Funksjonen skal returnere et heltall (dvs. i titallssystemet) som tilsvarer det binære tallet angitt med tekststrengen $binary$. Oppgaven \textbf{skal ikke løses} ved hjelp av innebygde funksjoner for å oversette binærtall til heltall. 

\end{frame}

\begin{frame}[fragile]
    \frametitle{Programmering 2016 kont 2b}

\begin{python}
def bin_to_dec(binary): 
    decimal = 0
    reversed_binary = binary[::-1]
    for i in range(len(reversed_binary)):
        decimal += int(reversed_binary[i]) * 2**i 
    return decimal
\end{python}

\end{frame}

\begin{frame}
    \frametitle{Programmering 2016 kont 2c}

    Lag funksjonen $dec_to_char$ som har en inn-parameter $dec$, som er et heltall med verdi mellom 0 og 31. Funksjonen skal returnere et tegn eller en bokstav avhengig av verdien av $dec$. 

    \begin{itemize}
        \item Hvis $dec$ har verdien 0 skal tegnet for " " (mellomrom) returneres
        \item Hvis $dec$ har verdien 1 skal tegnet for "," (komma) returneres
        \item Hvis $dec$ har verdien 2 skal tegnet for "." (punktum) returneres
        \item Hvis $dec$ har en verdi mellom 3 og 31 skal en stor bokstav i det norske alfabetet returneres, der $dec=3$ gir bokstaven "A", $dec=4$ gir bokstaven "B", helt opp til $dec=31$ som gir bokstaven "Å". 
        \item For alle andre verdier av $dec$ skal funksjonen returnere en tom streng. 
    \end{itemize}

\end{frame}

\begin{frame}[fragile]
    \frametitle{Programmering 2016 kont 2c}

    Antar denne konstanten eksisterer. 
    $DTC = " ,.ABCDEFGHIJKLMNOPQRSTUVWXYZÆØÅ"$

\begin{python}
def dec_to_char(dec):
    if dec < len(DTC): 
        return DTC[dec]
    else: 
        return ""
\end{python}

\end{frame}

\begin{frame}
    \frametitle{Programmering 2016 kont 2d}

    Lag funksjonen $bin_to_txt$ som har en inn-parameter $binstring$, som er en tekststreng av ukjent lengde bestående av binære tall (tekststreng ed nuller og enere). Funksjonen skal returnere en tekststreng bestående av bokstaver og tegn som er kodet i henhold til oppgave 2c der hvert tegn er representert med 5 bit. Inn-parameteren $binstring$ vil alltid være et multiplum av fem siffer. 

\end{frame}

\begin{frame}[fragile]
    \frametitle{Programmering 2016 kont 2d}

\begin{python}
def bin_to_txt(binstring): 
    txt = ""
    for i in range(0, len(binstring), 5):
        decimal = bin_to_dec(binstring[i:i+5])
        txt    += dec_to_char(decimal)
    return txt
\end{python}

\end{frame}

\begin{frame}[fragile]
    \frametitle{Programmering 2016 kont 2e}

    Lag funksjonen $main$, uten parametere. Funksjonen skal gjøre følgende: 

    \begin{enumerate}
        \item Skrive ut teksten "Binary-to-text converter" til skjerm 
        \item Spørre brukeren om navn på fil det skal lastes fra (tekstfil som inneholder binære tall) og ta vare på filnavnet i variabelen $b_file$. 
        \item Oversette tekststrengen av binære tall til tekst med bokstaver og tegn og lagre innholdet i variabelen $txt$. 
        \item Spørre brukeren om navnet på fil som resultatet skal lagres til og ta vare på filnavnet i variabelen $t_file$. 
        \item Skrive innholdet av variabelen $txt$ til fila med filnavn angitt i variabelen $t_file$. 
        \item Skrive ut til brukeren at: "$b_file$ has been converted and saved to $t_file$".
    \end{enumerate}

    Hvis funksjonen får problemer med å skrive til fila, skal følgende feilmelding skrives: "Error: Could not write to file $t_file$". 

\end{frame}

\begin{frame}[fragile]
    \frametitle{Programmering 2016 kont 2e}

\begin{python}
def main():
    print("Binary-to-text converter")
    b_file      = input("Name of binary file to load from: ")
    b_string    = load_bin(b_file)
    txt         = bin_to_txt(b_string)
    t_file      = input("Name of text file to save to: ")
    try:
        f = open(t_file, "w")
        f.write(txt)
        f.close()
        print(b_file, "has been converted and saved to", t_file)
    except:
        print("Error: Could not write to file", t_file)
\end{python}

\end{frame}

\begin{frame}
    \frametitle{Programmering 2013 2a}

    Lag funksjonen $yatzy$. Den skal ha 5 innparametere, kalt t1, t2, t3, t4 og t5. Innparamtrene representerer 5 tall mellom 1 og 6 (5 terninger). 

    Funksjonen skal returnere ei liste som inneholder de 5 tallene i sortert rekkefølge, eller en feilmelding hvis en av tallene er større enn 6 eller mindre enn 1. 

\end{frame}

\begin{frame}[fragile]
    \frametitle{Programmering 2013 2a}

\begin{python}
def yatzy(t1, t2, t3, t4, t5):
    l = [t1, t2, t3, t4, t5]
    for i in range(len(l)): 
        for j in range(len(l) - i - 1):
            if l[j] > 6 or l[j + 1] > 6: 
                return "Ikke bruk input stoerre enn 6!"
            if l[j] < 1 or l[j + 1] < 1: 
                return "Ikke bruk input mindre enn 1!"

            if l[j] > l[j + 1]: 
                l[j], l[j + 1] = l[j + 1], l[j]
    return l 
\end{python}

\end{frame}

\begin{frame}
    \frametitle{Programmering 2013 2b}

    Lag funksjonen $maxi_yatzy$. Den skal ta inn en liste med 5 eller 6 tall, og den skal returnere en skriftlig melding til brukeren som sier hvor mange terninger som ble kastet, hvilken verdi det var flest av, og hvor mange like det var av den verdien. Hvis det blir "uavgjort mellom to tall" brukes det høyeste tallet. 

\end{frame}

\begin{frame}[fragile]
    \frametitle{Programmering 2013 2b}

\begin{python}
    def maxi_yatzy(lst):
    n = len(lst)
    most_val = None
    most_num = 0

    for base_dice in range(1, 6 + 1):
        num = 0
        for dice in lst: 
            if dice == base_dice: 
                num += 1 
        if num >= most_num: 
            most_val = base_dice 
            most_num = num 
    return "Du kastet " + str(n) + " terninger og fikk flest " + str(most_val)  + " (" + str(most_num) + " like)."
\end{python}

\end{frame}




\end{document}
