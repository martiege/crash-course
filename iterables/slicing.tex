\begin{frame}
    \frametitle{Slicing}

    Slicing er en måte å indeksere en bestemt del av en iterable, hovedsakelig lister, strenger og tupler.

\end{frame}

\begin{frame}[fragile]
    \frametitle{Slicing}

    Slicing er en måte å indeksere en bestemt del av en iterable, hovedsakelig lister, strenger og tupler. Vi bruker firkantparenteser, siden vi aksesserer, og samme syntaks som en \textit{range} for å bestemme hva vi vil aksessere.

\begin{python}
iterable[fra_og_med_start:til_men_ikke_med_slutt:steg]
\end{python}

\end{frame}

\begin{frame}[fragile]
    \frametitle{Slicing}

    Dersom vi vil starte på start, \textit{trenger man ikke skrive noe før første kolon}.

\begin{python}
iterable[:til_men_ikke_med_slutt:steg]
\end{python}

    Dersom vi vil slutte på, inklusiv, slutten, \textit{trenger man ikke skrive noe mellom første og andre kolon}. 

\begin{python}
iterable[fra_og_med_start::steg]
\end{python}

    Dersom vi vil bruke 1 som \textit{steg}, \textit{trenger man ikke skrive noe etter siste kolon}, og man \textit{trenger ikke ha med siste kolon heller}. 

\begin{python}
iterable[fra_og_med_start:til_men_ikke_med_slutt:]
iterable[fra_og_med_start:til_men_ikke_med_slutt]
\end{python}

\end{frame}

\begin{frame}[fragile]
    \frametitle{Slicing}

    Grunnen til at slicing går fra og med, til men ikke med, er at da kan man dele opp en iterable med samme indeks. 

\begin{python}
lst[:ind] + lst[ind:] == lst # -> True
\end{python}

\end{frame}

\begin{frame}[fragile]
    \frametitle{Slicing}

    Dette kan f.eks. brukes for kjapp kopiering eller reversering av lister.

    Slicing returnerer \textbf{nye} objekter, i motsetning til f.eks. \textit{lst.reverse()}. 

\begin{python}
kopi = lst[:]
rev  = lst[::-1]
\end{python}

\end{frame}

\begin{frame}[fragile]
    \frametitle{Slicing}

    Vi kan også bruke slicing til å korte ned denne koden fra tidligere. 

\begin{python}
def insert_string(original, index, new_string):
    # antar at index er valid
    resultat = "" # starter med tom streng 
    for i in range(len(original)): 
        if i == index: 
            resultat += new_string 
        # om vi plusser foran eller bak new_string 
        # bestemmer om vi inserter foran eller bak. 
        resultat += original[i] 
    return resultat
\end{python}

\end{frame}

\begin{frame}[fragile]
    \frametitle{Slicing}

    Vi kan også bruke slicing til å korte ned denne koden fra tidligere. 

\begin{python}
def insert_string(original, index, new_string):
    # antar at index er valid
    return original[:index] + new_string + original[index:]
\end{python}

\end{frame}






































