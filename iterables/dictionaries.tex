
\begin{frame}
    \frametitle{Dictionaries}

    Dictionaries er kanskje den nyttigeste datastrukturen i Python, og en av de mest brukte. 

\end{frame}

\begin{frame}
    \frametitle{Dictionaries}

    Dictionaries er kanskje den nyttigeste datastrukturen i Python, og en av de mest brukte. Det som er viktig å huske på er at en dictionary \textit{kobler} sammen ulike verdier. 

\end{frame}

\begin{frame}
    \frametitle{Dictionaries}

    Dictionaries er kanskje den nyttigeste datastrukturen i Python, og en av de mest brukte. Det som er viktig å huske på er at en dictionary \textit{kobler} sammen ulike verdier. 

    Vi kobler fra en \textit{nøkkel} til en \textit{verdi}.

\end{frame}

\begin{frame}[fragile]
    \frametitle{Dictionaries}

    Syntaksen fungerer i prinsipp på samme måte som en liste. Vi bruker firkantparenteser til å aksessere \textit{verdien} som ligger på en \textit{nøkkel}.

\begin{python}
dictionary = {"katt": 1, "hund": 2}
print(dictionary["katt"]) # gir 1
\end{python}

\end{frame}

\begin{frame}[fragile]
    \frametitle{Dictionaries}

    Kan forenkle problemer som kan være vanskelige med lister. Eksempel: Telle opp antall bokstaver i en streng. 

\begin{python}
streng = "heihei"
# alle bokstaver fra a til z
bokstaver = [chr(bokstav) for bokstav in range(ord("a"), ord("z") + 1, 1)]
oppteller = [0] * len(bokstaver)
for i in range(len(bokstaver)): 
    oppteller[i] = streng.count(bokstaver[i])
\end{python}

\end{frame}

\begin{frame}[fragile]
    \frametitle{Dictionaries}

    Kan forenkle problemer som kan være vanskelige med lister. Eksempel: Telle opp antall bokstaver i en streng. 

    Kan være vanskelig å håndtere dette, siden vi har to separate lister. Finnes selvsagt andre løsninger, men for å få en direkte kobling mellom bokstav og antall er det enklest med dictionary. 

\begin{python}
streng = "heihei"
bokstaver_til_antall = {}
for bokstav in streng: 
    antall = bokstaver_til_antall.setdefault(bokstav, 0)
    bokstaver_til_antall[bokstav] = antall + 1
\end{python}

\end{frame}

\begin{frame}
    \frametitle{Dictionaries}

    Metoder

    \begin{table}[]
        \begin{tabular}{|l|l|}
        \hline
        Metode   & Eksempel                      \\ \hline
        clear    & dictionary.clear()            \\ \hline
        copy     & kopi = dictionary.copy()      \\ \hline
        fromkeys & dict.fromkeys(keys, values)   \\ \hline
        get      & value = dictionary.get(key)   \\ \hline
        items    & t\_list = dictionary.items()  \\ \hline
        values   & verdier = dictionary.values() \\ \hline
        \end{tabular}
    \end{table}

\end{frame}


\begin{frame}
    \frametitle{Dictionaries}

    Metoder (Cont.)

    \begin{table}[]
        \begin{tabular}{|l|l|}
        \hline
        Metode     & Eksempel                                    \\ \hline
        keys       & nøkler = dictionary.keys()                  \\ \hline
        pop        & dictionary.pop(key)                         \\ \hline
        popitem    & dictionary.popitem()                        \\ \hline
        setdefault & value = dictionary.setdefault(key, default) \\ \hline
        update     & dictionary.update(iterable)                 \\ \hline
        update     & dictionary.update(other\_dict)              \\ \hline
        \end{tabular}
    \end{table}

\end{frame}

\begin{frame}[fragile]
    \frametitle{Dictionaries}

    Kombineres godt med for-løkker. Avhengig av hva du trenger, har vi flere metoder for å iterere oss gjennom en dictionary. 

\begin{python}
dictionary = {"hello": "world", 12: 9}
\end{python}

\end{frame}

\begin{frame}[fragile]
    \frametitle{Dictionaries}

    Kombineres godt med for-løkker. Avhengig av hva du trenger, har vi flere metoder for å iterere oss gjennom en dictionary. 

    Dersom vi kun er interessert i verdiene, og ikke nøklene (ofte hvis noe kun skal printes ut): 

\begin{python}
dictionary = {"hello": "world", 12: 9}
for value in dictionary.values(): 
    print(value)
\end{python}

\end{frame}

\begin{frame}[fragile]
    \frametitle{Dictionaries}

    Kombineres godt med for-løkker. Avhengig av hva du trenger, har vi flere metoder for å iterere oss gjennom en dictionary. 

    Dersom vi kun er interessert i nøklene, og ikke verdiene, eller ønsker å aksessere verdiene med firkantparenteser: 

\begin{python}
dictionary = {"hello": "world", 12: 9}
for key in dictionary.keys(): 
    print(key, dictionary[key])
\end{python}

\end{frame}

\begin{frame}[fragile]
    \frametitle{Dictionaries}

    Kombineres godt med for-løkker. Avhengig av hva du trenger, har vi flere metoder for å iterere oss gjennom en dictionary. 

    Dersom vi er interessert i både verdi og nøkler, som vi kanskje oftest er: 

\begin{python}
dictionary = {"hello": "world", 12: 9}
for key, value in dictionary.items(): 
    print(key, value)
\end{python}

\end{frame}

\begin{frame}[fragile]
    \frametitle{Dictionaries}

    Husk: det er lett å gå fra nøkkel til verdi, men vanskelig å gå fra verdi til nøkkel: 

\begin{python}
# value to key
dictionary = {"hello": "world", 12: 9}
key = "hello"
value = dictionary[key]
\end{python}

\end{frame}

\begin{frame}[fragile]
    \frametitle{Dictionaries}

    Husk: det er lett å gå fra nøkkel til verdi, men vanskelig å gå fra verdi til nøkkel: 

\begin{python}
# value to key
dictionary = {"hello": "world", 12: 9}
key = None # ingen key enda, setter den til None
value = "hello"
for current_key, current_value in dictionary.items(): 
    if current_value == value: 
        key = current_key
        break # bare en key pr. value
\end{python}

\end{frame}












