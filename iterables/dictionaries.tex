
\begin{frame}
    \frametitle{Dictionaries}

    Dictionaries er kanskje den nyttigeste datastrukturen i Python, og en av de mest brukte. 

\end{frame}

\begin{frame}
    \frametitle{Dictionaries}

    Dictionaries er kanskje den nyttigeste datastrukturen i Python, og en av de mest brukte. Det som er viktig å huske på er at en dictionary \textit{kobler} sammen ulike verdier. 

\end{frame}

\begin{frame}
    \frametitle{Dictionaries}

    Dictionaries er kanskje den nyttigeste datastrukturen i Python, og en av de mest brukte. Det som er viktig å huske på er at en dictionary \textit{kobler} sammen ulike verdier. 

    Vi kobler fra en \textit{nøkkel} til en \textit{verdi}.

\end{frame}

\begin{frame}[fragile]
    \frametitle{Dictionaries}

    Syntaksen fungerer i prinsipp på samme måte som en liste. Vi bruker firkantparenteser til å aksessere \textit{verdien} som ligger på en \textit{nøkkel}.

\begin{python}
dictionary = {"katt": 1, "hund": 2}
print(dictionary["katt"]) # gir 1
\end{python}

\end{frame}

\begin{frame}[fragile]
    \frametitle{Dictionaries}

    Kan forenkle problemer som kan være vanskelige med lister. Eksempel: Telle opp antall bokstaver i en streng. 

\begin{python}
streng = "heihei"
# alle bokstaver fra a til z
bokstaver = [chr(bokstav) for bokstav in range(ord("a"), ord("z") + 1, 1)]
oppteller = [0] * len(bokstaver)
for i in range(len(bokstaver)): 
    oppteller[i] = streng.count(bokstaver[i])
\end{python}

\end{frame}

\begin{frame}[fragile]
    \frametitle{Dictionaries}

    Kan forenkle problemer som kan være vanskelige med lister. Eksempel: Telle opp antall bokstaver i en streng. 

    Kan være vanskelig å håndtere dette, siden vi har to separate lister. Finnes selvsagt andre løsninger, men for å få en direkte kobling mellom bokstav og antall er det enklest med dictionary. 

\begin{python}
streng = "heihei"
bokstaver_til_antall = {}
for bokstav in streng: 
    antall = bokstaver_til_antall.setdefault(bokstav, 0)
    bokstaver_til_antall[bokstav] = antall + 1
\end{python}

\end{frame}
