
\begin{frame}
    \frametitle{Input og output - Skrive til filer}

    Igjen brukes \textit{open}-funksjonen. Viktig å holde tunga rett i munnen, siden det er mye som kan gå galt her.

\end{frame}


\begin{frame}[fragile]
    \frametitle{Input og output - Skrive til filer}

    Igjen brukes \textit{open}-funksjonen. Viktig å holde tunga rett i munnen, siden det er mye som kan gå galt her.

    Her er det tre nye moduser, ikke bare \textit{"r"}, som tidligere. 

    \begin{itemize}
        \item "w": Write, standard skriving. Dersom fila eksisterer fra før av, slettes den opprinnelige fila. 
        \item "a": Append, legger til tekst på slutten av fila. Dersom den ikke eksisterer, opprettes den. 
        \item "x": Exclusive write, vil feile dersom fila allerede eksisterer. Kan bare brukes til å opprette nye filer. 
    \end{itemize}

\end{frame}

\begin{frame}[fragile]
    \frametitle{Input og output - Skrive til filer}

    Igjen brukes \textit{open}-funksjonen. Viktig å holde tunga rett i munnen, siden det er mye som kan gå galt her.

    Her er det tre nye moduser, ikke bare \textit{"r"}, som tidligere. 

    \begin{itemize}
        \item "w": Write, standard skriving. Dersom fila eksisterer fra før av, slettes den opprinnelige fila. 
        \item "a": Append, legger til tekst på slutten av fila. Dersom den ikke eksisterer, opprettes den. 
        \item "x": Exclusive write, vil feile dersom fila allerede eksisterer. Kan bare brukes til å opprette nye filer. 
    \end{itemize}

    Det finnes også en modus "+", som tillater oppdatering (read og write samtidig), men er oftest bedre å gjøre dette i to steg. 

\end{frame}

\begin{frame}[fragile]
    \frametitle{Input og output - Skrive til filer}

    Igjen brukes \textit{open}-funksjonen. Viktig å holde tunga rett i munnen, siden det er mye som kan gå galt her.

    Tilsvarende som med å lese fra filer, kan filer også åpnes som binærfiler. 

\end{frame}

\begin{frame}[fragile]
    \frametitle{Input og output - Skrive til filer}

    Igjen brukes \textit{open}-funksjonen. Viktig å holde tunga rett i munnen, siden det er mye som kan gå galt her.

    Tilsvarende som med å lese fra filer, kan filer også åpnes som binærfiler. 

\end{frame}

\begin{frame}[fragile]
    \frametitle{Input og output - Skrive til filer}

    Skriving til fil er dermed ganske rett frem. Man skriver fra \textit{der man er} i fila. 

\begin{python}
f = open("test.txt", "w")
f.write("Hello\n")
f.write("World!\n")
f.close()

g = open("test.txt", "a")
g.write("New\n")
g.write("lines!\n")
g.close()
\end{python}

\end{frame}


