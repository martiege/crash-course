
\begin{frame}[fragile]
    \frametitle{Datatyper}

    Variabler har bestemte typer, men kan byttes med tilordningsoperatoren. 

\begin{python}
a = "hei" # type er str
a = 1 # type er int
\end{python}

\end{frame}

\begin{frame}[fragile]
    \frametitle{Datatyper}

    Noen aritmetiske operatorer fungerer for andre typer enn tall, men ikke alltid. 

\begin{python}
liste1 = [1, 2, 3]
liste2 = [4, 5]
liste3 = liste1 + liste2

liste4 = liste3 * 2
\end{python}

\end{frame}


\begin{frame}[fragile]
    \frametitle{Datatyper}

    Eksempler på datatyper: 

    \begin{itemize}
        \item int: heltall, 1023913213, 0, -1
        \item float: flyttall, 0.1, 1.2e-10, 3.1415e6
        \item str: streng, "hei", "sacasdcasdc"
        \item list: liste, [1, 2, 3], ["hei", 1.3e2]
        \item tuple: tupler, (1, 2, 3), ("hei", 1.3e2)
    \end{itemize}

\end{frame}

\begin{frame}
    \frametitle{Datatyper}

    Funksjoner for å konvertere mellom datatyper. De gjør forskjellige ting for forskjellig input-typer, og fungerer ikke for alle. 

    \begin{table}[]
        \begin{tabular}{|l|l|l|}
        \hline
        Funksjon & Eksempel           & Resultat    \\ \hline
        bin()    & bin(92)            & '0b1011100' \\ \hline
        bool()   & bool(12), bool(0)  & True, False \\ \hline
        chr()    & chr(97)            & 'a'         \\ \hline
        ord()    & ord('a')           & 97          \\ \hline
        float()  & float(1)           & 1.0         \\ \hline
        int()    & int(2.6), int("2") & 2, 2        \\ \hline
        list()   & list(range(2))     & {[}0, 1{]}  \\ \hline
        str()    & str(100)           & '100'       \\ \hline
        \end{tabular}
    \end{table}

\end{frame}

\begin{frame}[fragile]
    \frametitle{Datatyper}

    Type gir hvilken datatype en variabel er, dermed kan vi sjekke med en if. Kan også bruke \textit{isinstance}

\begin{python}
a = 1.23
if type(a) == float:
    print("a er en float")
if isinstance(a, float):
    print("a er fortsatt en float")
\end{python}

\end{frame}

\begin{frame}
    \frametitle{Datatyper}

    Dette kan ofte være nyttig i funksjoner for å gjøre forskjellige ting for forskjellige parametere. 

\end{frame}
