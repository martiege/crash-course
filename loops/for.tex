
\begin{frame}
    \frametitle{Løkker - for-løkken}

    for-løkker brukes når man har en bestemt \textit{iterable} man ønsker å iterere gjennom. 

\end{frame}


\begin{frame}[fragile]
    \frametitle{Løkker - for-løkken}

    for-løkker brukes når man har en bestemt \textit{iterable} man ønsker å iterere gjennom. Dette gjelder spesielt de vi allerede har gått gjennom, i form av \textit{lister} og \textit{tupler}. 

\end{frame}

\begin{frame}[fragile]
    \frametitle{Løkker - for-løkken}

    for-løkker brukes når man har en bestemt \textit{iterable} man ønsker å iterere gjennom. Dette gjelder spesielt de vi allerede har gått gjennom, i form av \textit{lister} og \textit{tupler}. 

    Disse kan defineres \textit{inline}

\begin{python}
for word in ("hei", "hei", "alle", "sammen"): 
    print(word)
for word in ["hei", "hei", "alle", "sammen"]: 
    print(word)
\end{python}

\end{frame}

\begin{frame}[fragile]
    \frametitle{Løkker - for-løkken}

    for-løkker brukes når man har en bestemt \textit{iterable} man ønsker å iterere gjennom. Dette gjelder spesielt de vi allerede har gått gjennom, i form av \textit{lister} og \textit{tupler}. 

    Eller på forhånd. Da kan de gjenbrukes senere, eller redefineres før. 

\begin{python}
tpl = ("hei", "hei", "alle", "sammen")
lst = ["hei", "hei", "alle", "sammen"]
for word in tpl: 
    print(word)
for word in lst: 
    print(word)
\end{python}

\end{frame}

\begin{frame}
    \frametitle{Løkker - for-løkken}

    Det er hovedsakelig to måter å \textit{iterere} gjennom en liste. 

\end{frame}

\begin{frame}[fragile]
    \frametitle{Løkker - for-løkken}

    Det er hovedsakelig to måter å \textit{iterere} gjennom en liste. 

    Element-vis. Man henter ut hvert \textit{element} i listen.

\begin{python}
lst = [1, 2, 3, "hei", (1, 2, 3)]
for element in lst: 
    print(element)
\end{python}

\end{frame}

\begin{frame}[fragile]
    \frametitle{Løkker - for-løkken}

    Det er hovedsakelig to måter å \textit{iterere} gjennom en liste. 

    Indeks-vis. Man henter ut \textit{plasseringen} til hvert element i listen. 

\begin{python}
lst = [1, 2, 3, "hei", (1, 2, 3)]
for i in range(len(lst)): 
    print(lst[i])
\end{python}

\end{frame}

\begin{frame}[fragile]
    \frametitle{Løkker - for-løkken}

    Det er hovedsakelig to måter å \textit{iterere} gjennom en liste. 

    Indeks-vis. Man henter ut \textit{plasseringen} til hvert element i listen. 

    Viktig: hvis vi itererer element-vis gjennom en liste, vet vi egentlig ingenting om plasseringen, og hvis vi trenger indeksen, er det best å gå gjennom indeks-vis. 

\begin{python}
lst = [1, 2, 3, "hei", (1, 2, 3)]
for i in range(len(lst)): 
    print(lst[i])
\end{python}

\end{frame}

\begin{frame}[fragile]
    \frametitle{Løkker - for-løkken}

    Det er hovedsakelig to måter å \textit{iterere} gjennom en liste. 

    Kan bruke \textit{enumerate} for å få ut begge med en gang. 

\begin{python}
lst = [1, 2, 3, "hei", (1, 2, 3)]
for i, element in enumerate(lst): 
    # lst[i] == element -> True
    print(i, lst[i], element)
\end{python}

\end{frame}

\begin{frame}[fragile]
    \frametitle{Løkker - for-løkken}

    Det er hovedsakelig to måter å \textit{iterere} gjennom en liste. 

    Kan bruke \textit{enumerate} for å få ut begge med en gang. Sjeldent dette ikke kan løses med bare indekser, og tilordning. 

\begin{python}
lst = [1, 2, 3, "hei", (1, 2, 3)]

for i, element in enumerate(lst): 
    # lst[i] == element -> True
    print(i, lst[i], element)

for i in range(len(lst)):
    element = lst[i]
    # lst[i] == element -> True
    print(i, lst[i], element)
\end{python}

\end{frame}

\begin{frame}[fragile]
    \frametitle{Løkker - for-løkken}

    Løkker kan enkelt skrives inne i hverandre. 

\begin{python}
def antall_like(lst1, lst2):
    res = 0
    for x in lst1: 
        for y in lst2:
            if x == y: 
                res += 1
    
    return res
\end{python}

\end{frame}

\begin{frame}[fragile]
    \frametitle{Løkker - for-løkken}

    Løkker kan enkelt skrives inne i hverandre. Går gjennom hvert element i \textit{lst1}, sjekker hvert element i \textit{lst2} og ser om jeg finner det. 

\begin{python}
def antall_like(lst1, lst2):
    res = 0
    for x in lst1: 
        for y in lst2:
            if x == y: 
                res += 1
    
    return res
\end{python}

\end{frame}

\begin{frame}[fragile]
    \frametitle{Løkker - for-løkken}

    Løkker kan enkelt skrives inne i hverandre. Går gjennom hvert element i \textit{lst1}, sjekker hvert element i \textit{lst2} og ser om jeg finner det. 

    Her trenger vi ikke gå gjennom med indekser (range), siden vi trenger ikke vite noe om posisjon. 

\begin{python}
def antall_like(lst1, lst2):
    res = 0
    for x in lst1: 
        for y in lst2:
            if x == y: 
                res += 1
    
    return res
\end{python}

\end{frame}

\begin{frame}[fragile]
    \frametitle{Løkker - for-løkken}

    Det kan se vanskelig ut å analysere når det er flere løkker. 

\begin{python}
def antall_like(lst1, lst2):
    res = 0
    for x in lst1: 
        for y in lst2:
            if x == y: 
                res += 1
    
    return res
\end{python}

\end{frame}

\begin{frame}[fragile]
    \frametitle{Løkker - for-løkken}

    Det kan se vanskelig ut å analysere når det er flere løkker. Det er enklest å begynne å analysere \textit{innenfra og ut}.

\begin{python}
def antall_like(lst1, lst2):
    res = 0
    for x in lst1: 
        for y in lst2:
            if x == y: 
                res += 1
    
    return res
\end{python}

\end{frame}

\begin{frame}[fragile]
    \frametitle{Løkker - for-løkken}

    Det kan se vanskelig ut å analysere når det er flere løkker. Det er enklest å begynne å analysere \textit{innenfra og ut}. Dette gjelder også når vi skal designe kode som bruker kode inni kode. 

\begin{python}
def antall_like(lst1, lst2):
    res = 0
    for x in lst1: 
        for y in lst2:
            if x == y: 
                res += 1
    
    return res
\end{python}

\end{frame}

\begin{frame}[fragile]
    \frametitle{Løkker - for-løkken}

    Vi antar at \textit{x} og \textit{y} er bestemte verdier (f.eks. \textit{x} = 1, \textit{y} = 2). Hva er det vi vil gjøre med denne bestemte verdien? 

\begin{python}
def antall_like(lst1, lst2):
    res = 0
    for x in lst1: 
        for y in lst2:
            if x == y: 
                res += 1
    
    return res
\end{python}

\end{frame}

\begin{frame}[fragile]
    \frametitle{Løkker - for-løkken}

    Vi antar at \textit{x} og \textit{y} er bestemte verdier (f.eks. \textit{x} = 1, \textit{y} = 2). Hva er det vi vil gjøre med denne bestemte verdien? 

    Hvis disse er like, øker vi \textit{telleren} med 1. 

\begin{python}
def antall_like(lst1, lst2):
    res = 0
    for x in lst1: 
        for y in lst2:
            if x == y: 
                res += 1
    
    return res
\end{python}

\end{frame}

\begin{frame}[fragile]
    \frametitle{Løkker - for-løkken}

    Vi antar at \textit{x} og \textit{y} er bestemte verdier (f.eks. \textit{x} = 1, \textit{y} = 2). Hva er det vi vil gjøre med denne bestemte verdien? 

    Hvis disse er like, øker vi \textit{telleren} med 1. 

    Nå er den \textit{innerste} delen av koden abstrahert bort. Hva er neste steg? 

\begin{python}
def antall_like(lst1, lst2):
    res = 0
    for x in lst1: 
        for y in lst2:
            if x == y: 
                res += 1
    
    return res
\end{python}

\end{frame}

\begin{frame}[fragile]
    \frametitle{Løkker - for-løkken}

    Vi antar at \textit{x} er en bestemt verdi (f.eks. \textit{x} = 1). 

    Vi vet hva som skjer for en bestemt \textit{y}, så det kan generaliseres. Vi vet hva \textit{x} er, og vi vet hva vi skal gjøre hvis vi har en \textit{x} og en \textit{y}. Så vi går gjennom alle mulige \textit{y}-er. 

\begin{python}
def antall_like(lst1, lst2):
    res = 0
    for x in lst1: 
        for y in lst2:
            if x == y: 
                res += 1
    
    return res
\end{python}

\end{frame}

\begin{frame}[fragile]
    \frametitle{Løkker - for-løkken}

    Nå trenger vi ikke lenger å se på en bestemt verdi for \textit{x}, siden vi vet hva som skjer med en (den sammenlignes med hver mulige \textit{y}), kan den også generaliseres. 

    For hver mulige \textit{x}, sjekk hver mulige \textit{y}. Dersom \textit{x} == \textit{y}, så øker vi telleren med en.

\begin{python}
def antall_like(lst1, lst2):
    res = 0
    for x in lst1: 
        for y in lst2:
            if x == y: 
                res += 1
    
    return res
\end{python}

\end{frame}

\begin{frame}[fragile]
    \frametitle{Løkker - for-løkken}

    Strenger og tupler kan behandles ganske likt, siden begge er ikke-muterbare. F.eks., dersom vi vil endre noe, så må vi bygge disse strukturene på nytt. 

    Dette kan vi gjøre litt enklere senere med slicing. 

\begin{python}
def insert_string(original, index, new_string):
    # antar at index er valid
    resultat = "" # starter med tom streng 
    for i in range(len(original)): 
        if i == index: 
            resultat += new_string 
        # om vi plusser foran eller bak new_string 
        # bestemmer om vi inserter foran eller bak. 
        resultat += original[i] 
    return resultat
\end{python}

\end{frame}












