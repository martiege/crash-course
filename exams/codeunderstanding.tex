
\begin{frame}
    \frametitle{Kodeforståelse}

    Viktig å lese oppgaveteksten. 
    
    Ofte er det spørsmål om både hva som returneres for en bestemt input-verdi, i tillegg til å beskrive funksjonen med en setning. 

\end{frame}


\begin{frame}
    \frametitle{Kodeforståelse}

    Viktig å lese oppgaveteksten. 
    
    Ofte er det spørsmål om både hva som returneres for en bestemt input-verdi, i tillegg til å beskrive funksjonen med en setning. 

    Det vil \textit{ikke} si at vi skriver hva som skjer, men heller hva som er poenget med funksjonen. 

\end{frame}

\begin{frame}[fragile]
    \frametitle{Kodeforståelse}

    \textbf{Feil svar:} Funksjonen setter \textit{r} = 1, og så lenge \textit{x} er større eller lik 1 ganges \textit{r} med \textit{x}, og \textit{x} minker med 1. 

\begin{python}
def f(x): 
    r = 1
    while x >= 1: 
        r *= x
        x -= 1
    return r
\end{python}

\end{frame}

\begin{frame}[fragile]
    \frametitle{Kodeforståelse}

    \textbf{Riktig(ere) svar:} Funksjonen beregner verdien $x!$, x fakultet.

\begin{python}
def f(x): 
    r = 1
    while x >= 1: 
        r *= x
        x -= 1
    return r
\end{python}

\end{frame}

\begin{frame}[fragile]
    \frametitle{Kodeforståelse 2016 2d}

    Hva blir skrevet ut på skjermen hvis koden vist under blir kjørt? 

    Forklar med en setning hva funksjonen $f$ gjør

\begin{python}
def f(x):
    y = 0
    while x > 0:
        y = y + x % 10
        x = int( x / 10 )
    if y >= 10:
        y = f( y )
    return y

print( f(32145) )
\end{python}

\end{frame}

\begin{frame}
    \frametitle{Kodeforståelse 2016 kont 4b}

    6

    $f$ regner ut den rekursive tverrsummen tallet som den får inn som parameter, dvs. fortsetter å ta tverrsum helt til vi får et ensifret tall. 

    Tverrsummen av 32145 blir 15, tverrsummen av 15 blir deretter 6. 

\end{frame}

\begin{frame}[fragile]
    \frametitle{Kodeforståelse 2016 kont 4b}

    Hva returneres hvis man kaller $f([[3, 5], [2, 4], [1, 3]])$ med koden nedenfor? 

    Forklar med \underline{en setning} hva funksjonen $f()$ gjør? 

\begin{python}
def f(b): 
    c=len(b[0])
    d=len(b)
    g = [[0 for row in range(d)]
        for col in range(c)]
    for e in range(0, c):
        for f in range(0, d): 
            g[e][f] = b[f][e]
    return g
\end{python}

\end{frame}

\begin{frame}
    \frametitle{Kodeforståelse 2016 kont 4b}

    [[3, 2, 1], [5, 4, 3]]

    Evt. ingenting ettersom det er feil med innrykk i oppgaveteksten. 
    Begge svar gir full pott!

    Funksjonen transponerer en matrise. 

\end{frame}

\begin{frame}[fragile]
    \frametitle{Kodeforståelse 2017 2a}

    Funksjonen $bin_search$ er ment til å skulle utføre binærsøk, men resulterer i feilmeldingen \textit{"IndexError: list index out of range"}. 

    I hvilken linje er feilen? 

    Hva skulle egentlig stått på den linjen for at funksjonen skal virke etter sin hensikt? 

\begin{python}
def bin_search(liste, verdi, imin, imax): 
    if (imax < imin): 
        return False
    else: 
        imid = (imin + imax)
        if verdi < liste[imid]: 
            return bin_search(liste, verdi, imin, imid-1)
        elif verdi > liste[imid]: 
            return bin_search(liste, verdi, imid+1, imax)
        else: 
            return imid
\end{python}

\end{frame}

\begin{frame}
    \frametitle{Kodeforståelse 2017 2a}

    Linje 5

    $imid = (imid + imax) / 2$

\end{frame}

\begin{frame}[fragile]
    \frametitle{Kodeforståelse 2018 sett 1 2a}

    Hva returneres ved funksjonskallet under? 
    $myst(((True and False) or (False and True)), ((False or True) and (not(not True))))$

\begin{python}
def myst(val1, val2):
    if (val1 and val2):
        return 1 
    elif (val1 and not val2): 
        return 2
    elif (not val1 and val2): 
        return 3 
    else: 
        return 4
\end{python}

\end{frame}

\begin{frame}
    \frametitle{Kodeforståelse 2017 2a}

    3

\end{frame}

\begin{frame}[fragile]
    \frametitle{Kodeforståelse 2018 kont 2e og f}

    Hva blir skrevet ut til skjerm når du kjører programmet vist under?

    $z = ((2, 0, 11, 8, 5), (14, 17, 13, 8, 0))$
    $print(myst3(z))$

    Beskriv med en setning hva koden i oppgave 2e gjør. 
    
\begin{python}
def myst3(a): 
	s  = ''
	for r in a: 
		for c in r: 
			s += chr(ord('A') + c)
	return s
\end{python}

\end{frame}

\begin{frame}
    \frametitle{Kodeforståelse 2018 kont 2e og f}

    \textbf{CALIFORNIA}
    Store bokstaver (pga. $ord('A')$) og ingen fnutter (pga. printing)

    Henter ut ett og ett tall fra et tuppel av tupler, og lager en sammenhengende tekst av store bokstaver hvor c angir hvor langt unna hver bokstav er fra A i alfabetet (ASCII- tabellen).

\end{frame}
