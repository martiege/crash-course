
\begin{frame}
    \frametitle{Funksjoner}

    Svært ofte ønsker man å gjenbruke deler av koden man skriver. 

\end{frame}


\begin{frame}
    \frametitle{Funksjoner}

    Svært ofte ønsker man å gjenbruke deler av koden man skriver. Det å printe / skrive til skjermen er egentlig en veldig kompleks prosess, men den har blitt \textit{abstrahert bort}. Abstraksjoner er veldig viktige for å løse oppgaver, både til eksamen og senere. 

\end{frame}

\begin{frame}
    \frametitle{Funksjoner}

    Et viktig prinsipp er det å \textit{kalle} en funksjon. Når man \textit{definerer} en funksjon, lager man bare en generell oppskrift for ubestemte parametere. 
    
    Når man kaller på en funksjon, så utfører man den oppskriften. Det vil si at vi gir det noen \textit{fysiske} verdier å jobbe med. 

\end{frame}
