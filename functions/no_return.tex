
\begin{frame}
    \frametitle{Funksjoner - Funksjoner uten retur-verdi}

    I mange tilfeller skal en funksjon gjøre noe enkelt som ikke krever at koden husker på endringen etter funksjonen har blitt kjørt. 

\end{frame}

\begin{frame}
    \frametitle{Funksjoner - Funksjoner uten retur-verdi}

    I mange tilfeller skal en funksjon gjøre noe enkelt som ikke krever at koden husker på endringen etter funksjonen har blitt kjørt. 

    Dette er ofte relatert til \textit{output}, som f.eks. printing og skrive til fil. 

\end{frame}

\begin{frame}[fragile]
    \frametitle{Funksjoner - Funksjoner uten retur-verdi}

    Funksjoner begynner med kodeordet \textit{def}, funksjonsnavnet, en parentes med alle parameterene funksjonen tar inn og \textbf{VIKTIG} kolon :. Det må også være minst en kodesnutt i funksjonen.

    Kanskje den enkleste funksjonen?

\begin{python}
def func():
    pass
\end{python}

\end{frame}

\begin{frame}[fragile]
    \frametitle{Funksjoner - Funksjoner uten retur-verdi}

    Eksempel med printing. 

\begin{python}
def print_resultater(res, tekst):
    print("Resultatet ble",res,sep=": ",end=" og ")
    print("dette er teksten: " + tekst)
\end{python}

\end{frame}

\begin{frame}[fragile]
    \frametitle{Funksjoner - Funksjoner uten retur-verdi}

    Vi har allerede jobbet med default-parametere. \textit{sep} og \textit{end} er default-verdier i \textit{print}-funksjonen, og med mindre de spesifiseres vil \textit{sep} alltid være " " (mellomrom) og \textit{end} alltid være "\textbackslash n" (ny linje). 

    Her bruker vi det til å bestemme om vi skal printe eller lagre til fil. 

\begin{python}
def output_resultater(res, file_path=""):
    if file_path != "":
        f = open(file_path, "w")
        f.write(res)
        f.close()
    else:
        print("Her er resultatet:", res)

output_resultater("hei") # printes
output_resultater("hei", "test.txt") # skrives til fil
output_resultater("hei", file_path="test.txt")
\end{python}

\end{frame}
