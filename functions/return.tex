
\begin{frame}
    \frametitle{Funksjoner - Funksjoner med retur-verdi}

    Dersom man ikke returnerer en verdi fra en funksjon, kan man ikke egentlig forvente at noe endres utenfor funksjonen vår. 

\end{frame}

\begin{frame}
    \frametitle{Funksjoner - Funksjoner med retur-verdi}

    Dersom man ikke returnerer en verdi fra en funksjon, kan man ikke egentlig forvente at noe endres utenfor funksjonen vår. 

    Dette er ikke helt korrekt, men kommer tilbake til dette senere. 

\end{frame}

\begin{frame}[fragile]
    \frametitle{Funksjoner - Funksjoner med retur-verdi}

    Det vi vil skal komme ut av funksjonen vår, det som skal returneres, skriver vi etter return. 

\begin{python}
def pluss(tall_1, tall_2):
    tall_sum = tall_1 + tall_2
    return tall_sum
\end{python}

\end{frame}

\begin{frame}[fragile]
    \frametitle{Funksjoner - Funksjoner med retur-verdi}

    \textbf{Viktig}: Når en verdi returneres fra en funksjon, så kan det tolkes som at vi \textit{bytter} ut funksjonskallet med det som beregnes i og returneres av funksjonen. 

\begin{python}
def pluss(tall_1, tall_2):
    tall_sum = tall_1 + tall_2
    return tall_sum

resultat = pluss(1, 2)
print(resultat)
\end{python}

\end{frame}

\begin{frame}[fragile]
    \frametitle{Funksjoner - Funksjoner med retur-verdi}

    \textbf{Viktig}: Når en verdi returneres, må den tilordnes, ellers glemmer programmet hva resultatet er. 

    I noen tilfeller trenger vi kanskje ikke resultatet til noe annet enn f.eks. printing, men ofte er det lurt å tilordne. 

\begin{python}
def pluss(tall_1, tall_2):
    tall_sum = tall_1 + tall_2
    return tall_sum

resultat = pluss(1, 2)
print(resultat)
\end{python}

\end{frame}

\begin{frame}[fragile]
    \frametitle{Funksjoner - Funksjoner med retur-verdi}

    Styrken med funksjoner er at de kan være ganske generelle, og spesifiseres kun når de kalles. 
    
    Sammen med løkker og lister er funksjoner noe av det viktigste med programmering. 

\end{frame}