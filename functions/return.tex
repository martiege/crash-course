
\begin{frame}
    \frametitle{Funksjoner - Funksjoner med retur-verdi}

    Dersom man ikke returnerer en verdi fra en funksjon, kan man ikke egentlig forvente at noe endres utenfor funksjonen vår. 

\end{frame}

\begin{frame}
    \frametitle{Funksjoner - Funksjoner med retur-verdi}

    Dersom man ikke returnerer en verdi fra en funksjon, kan man ikke egentlig forvente at noe endres utenfor funksjonen vår. 

    Dette er ikke helt korrekt, men kommer tilbake til dette senere. 

\end{frame}

\begin{frame}[fragile]
    \frametitle{Funksjoner - Funksjoner med retur-verdi}

    Det vi vil skal komme ut av funksjonen vår, det som skal returneres, skriver vi etter return. 

\begin{python}
def pluss(tall_1, tall_2):
    tall_sum = tall_1 + tall_2
    return tall_sum
\end{python}

\end{frame}

\begin{frame}[fragile]
    \frametitle{Funksjoner - Funksjoner med retur-verdi}

    \textbf{Viktig}: Når en verdi returneres fra en funksjon, så kan det tolkes som at vi \textit{bytter} ut funksjonskallet med 

\begin{python}
def pluss(tall_1, tall_2):
    tall_sum = tall_1 + tall_2
    return tall_sum
\end{python}

\end{frame}